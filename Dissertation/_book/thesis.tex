% From https://github.com/UWIT-IAM/UWThesis

\documentclass [11pt, proquest] {uwthesis}[2015/03/03]


% fix for pandoc 1.14
\providecommand{\tightlist}{%
  \setlength{\itemsep}{0pt}\setlength{\parskip}{0pt}}

\newtheorem{theorem}{Jibberish}

%% \bibliography{references}

\hyphenation{mar-gin-al-ia}

%
% ----- apply watermark to every page
% ----- change 'stamp' to 'nostamp'
%------ to omit watermark
%
\usepackage[nostamp]{draftwatermark}
% % Use the following to make modification
\SetWatermarkText{DRAFT}
\SetWatermarkLightness{0.95}

%% for the per mil symbol
\usepackage[nointegrals]{wasysym}

%% for copyright symbol
\usepackage{textcomp}

%% to allow to rotate pages to landscape
\usepackage{lscape}
%% to adjust table column width
\usepackage{tabularx}

% suppress bottom page numbers on first page of each chapter
% because they overlap with text
\usepackage{etoolbox}
\patchcmd{\chapter}{plain}{empty}{}{}

%% for more attractive tables
\usepackage{booktabs}
\usepackage{longtable}


\usepackage{graphicx}


% Double spacing, if you want it.
% \def\dsp{\def\baselinestretch{2.0}\large\normalsize}
% \dsp

% If the Grad. Division insists that the first paragraph of a section
% be indented (like the others), then include this line:
% \usepackage{indentfirst}

%%%%%%%%%%%%%%%%%%
% If you want to use "sections" to partition your thesis
% un-comment the following:
%
% \counterwithout{section}{chapter}
% \setsecnumdepth{subsubsection}
% \def\sectionmark#1{\markboth{#1}{#1}}
% \def\subsectionmark#1{\markboth{#1}{#1}}
% \renewcommand{\thesection}{\arabic{section}}
% \renewcommand{\thesubsection}{\thesection.\arabic{subsection}}
% \makeatletter
% \let\l@subsection\l@section
% \let\l@section\l@chapter
% \makeatother
%
% \renewcommand{\thetable}{\arabic{table}}
% \renewcommand{\thefigure}{\arabic{figure}}
%
%%%%%%%%%%%%%%%%%%


%% Stuff from https://github.com/suchow/Dissertate

% The following line would print the thesis in a postscript font

% \usepackage{natbib}
% \def\bibpreamble{\protect\addcontentsline{toc}{chapter}{Bibliography}}

\setcounter{tocdepth}{1} % Print the chapter and sections to the toc
% controls depth of table of contents (toc): 0 = chapter, 1 = section, 2 = subsection

\usepackage{biblatex}

\prelimpages

%% from thesisdown
% To pass between YAML and LaTeX the dollar signs are added by CII
\Title{Emily's Thesis Title}
\Author{Emily Pollock}
\Year{2021?}
\Program{Biological Anthropology}
\Chair{Steven M. Goodreau}{Title of my chair}{Biological Anthropology}
\Signature{person 1}
\Signature{person 2}
\Signature{person 3}

% commands and environments needed by pandoc snippets
% extracted from the output of `pandoc -s`
%% Make R markdown code chunks work
\usepackage{array}
\usepackage{amssymb,amsmath}
\usepackage{ifxetex,ifluatex}
\ifxetex
  \usepackage{fontspec,xltxtra,xunicode}
  \defaultfontfeatures{Mapping=tex-text,Scale=MatchLowercase}
\else
  \ifluatex
    \usepackage{fontspec}
    \defaultfontfeatures{Mapping=tex-text,Scale=MatchLowercase}
  \else
    \usepackage[utf8]{inputenc}
  \fi
\fi
\usepackage{color}
\usepackage{fancyvrb}
\DefineShortVerb[commandchars=\\\{\}]{\|}
\DefineVerbatimEnvironment{Highlighting}{Verbatim}{commandchars=\\\{\}}
% Add ',fontsize=\small' for more characters per line
\newenvironment{Shaded}{}{}
\newcommand{\KeywordTok}[1]{\textcolor[rgb]{0.00,0.44,0.13}{\textbf{{#1}}}}
\newcommand{\DataTypeTok}[1]{\textcolor[rgb]{0.56,0.13,0.00}{{#1}}}
\newcommand{\DecValTok}[1]{\textcolor[rgb]{0.25,0.63,0.44}{{#1}}}
\newcommand{\BaseNTok}[1]{\textcolor[rgb]{0.25,0.63,0.44}{{#1}}}
\newcommand{\FloatTok}[1]{\textcolor[rgb]{0.25,0.63,0.44}{{#1}}}
\newcommand{\CharTok}[1]{\textcolor[rgb]{0.25,0.44,0.63}{{#1}}}
\newcommand{\StringTok}[1]{\textcolor[rgb]{0.25,0.44,0.63}{{#1}}}
\newcommand{\CommentTok}[1]{\textcolor[rgb]{0.38,0.63,0.69}{\textit{{#1}}}}
\newcommand{\OtherTok}[1]{\textcolor[rgb]{0.00,0.44,0.13}{{#1}}}
\newcommand{\AlertTok}[1]{\textcolor[rgb]{1.00,0.00,0.00}{\textbf{{#1}}}}
\newcommand{\FunctionTok}[1]{\textcolor[rgb]{0.02,0.16,0.49}{{#1}}}
\newcommand{\RegionMarkerTok}[1]{{#1}}
\newcommand{\ErrorTok}[1]{\textcolor[rgb]{1.00,0.00,0.00}{\textbf{{#1}}}}
\newcommand{\NormalTok}[1]{{#1}}
\newcommand{\OperatorTok}[1]{\textcolor[rgb]{0.00,0.44,0.13}{\textbf{{#1}}}}
\newcommand{\BuiltInTok}[1]{\textcolor[rgb]{0.00,0.44,0.13}{\textbf{{#1}}}}
\newcommand{\ControlFlowTok}[1]{\textcolor[rgb]{0.00,0.44,0.13}{\textbf{{#1}}}}


\ifxetex
  \usepackage[setpagesize=false, % page size defined by xetex
              unicode=false, % unicode breaks when used with xetex
              xetex,
              colorlinks=true,
              linkcolor=blue]{hyperref}
\else
  \usepackage[unicode=true,
              colorlinks=true,
              linkcolor=blue]{hyperref}
\fi
\hypersetup{breaklinks=true, pdfborder={0 0 0}}
\setlength{\parindent}{0pt}
\setlength{\parskip}{6pt plus 2pt minus 1pt}
\setlength{\emergencystretch}{3em}  % prevent overfull lines
\setcounter{secnumdepth}{2} %% controls section numbering, e.g. 1 or 1.2, or 1.2.3

\begin{document}
\copyrightpage

\titlepage

\setcounter{page}{-1}
\abstract{``Here is my abstract''}

\tableofcontents
\listoffigures
\listoftables

\acknowledgments{``My acknowledgments''}

\dedication{\begin{center}``My dedication''\end{center}}

\textpages


\chapter*{Introduction}\label{introduction}
\addcontentsline{toc}{chapter}{Introduction}

Anthropologists have long recognized the importance of social
connections and behavioral variation among humans and our nonhuman
primate relatives. Indeed, the ability for us to participate in distinct
but potentially interlocking complex social networks has fueled our
evolution as a species and made our uniquely elaborate life possible.
Network analysis has often been utilized as a way to visually and
quantitatively represent these ties in order to understand their effects
on those connected to each other, from kinship, social support and
social capital, to the diffusion of information and transmission of
disease. These latter networks are crucially important to our
understanding of how human biosocial variation influences our health,
where the oft-beneficial complex social networks we maintain and
navigate every day can also put us at risk of exposure to infection.

transition to STIs

In order to understand the complex patterns by which sexually
transmitted infections (STIs) are transmitted throughout populations, we
first need to understand the behavior of human relationships and how
these behaviors generate the dynamic sexual network across which these
types of infections can spread.

This work is guided by the theoretical framework of the human ecology of
infectious disease, the investigation of how human behavior, social
patterns, and built environments interact with the broader pathogen
environment to influence our health. Of particular interest is not just
aggregate behavior, but also how variation in individual behavior
influences social patterns and alters the landscape through which
diseases spread, particularly as this variation relates to biological
age. Syndemic theory will also be used as a guide to understand how
variation in behaviors and patterns act synergistically to increase
vulnerability and exacerbate existing health disparities of certain
population subgroups (Singer et al., 2006).
\begin{itemize}
\tightlist
\item
  Whole other review on chlamydia goes here? Or in Chapter 3? I think
  some here may make sense to really motivate things. As you see, I've
  mentioned chlamydia enough times already that knowing the basic epi
  would be useful. It also really drives home the ``biological'' aspect
  of biological anthropology early on.
\end{itemize}
can I pull some stuff from my PAA abstract about age?
\begin{itemize}
\tightlist
\item
  transition to a history of the evolution of epidemic models (ie. from
  compartmental where everything is basically independent and
  exponential through to ERGMs where formation can be quite elaborate
  but we've never spent much time thinking about dissolution)
\end{itemize}
Mathematical models are quantitative representations of real-life
systems and the processes within these systems important to the outcome
of interest. This form of inquiry is particularly useful when classic
scientific experiments to understand disease spread or intervention
efficacy cannot be conducted for either practical or ethical reasons, or
when specific processes or parameter values in a system are unknown. In
these situations, we use mathematical modeling as an in-silico
laboratory to explore ideas and test hypotheses. Of course, the form and
complexity of these models are determined by a variety of factors
including the type of question that needs answering and the natural
history of the infection of interest, but many types of mathematical
models rely on similar underlying assumptions. Without diving too deep
into the history of epidemic modeling, here I give a brief overview of
the various model forms to highlight some key similarities and
differences.

Initial mathematical models for epidemics were deterministic and
compartmental in nature. They did not represent people individually,
rather they group them into homogenous compartments representing
specific states of interest, a portion of which transitioned between
compartments at each time step based on a rate. In the most basic
models, the compartments are usually ``susceptible'' and ``infected''
and the rate of transition from susceptible to infected depends on the
rate of contact between the groups and the size of the infected group
relative to the whole population. Additional complexity can be added by
adding more compartments or states, like breaking down the state of
susceptible and infected into demographic states like race or age
groups, adding compartments for vector populations like mosquitoes, or
by representing a more complex natural history of the pathogen by
including states for groups such as ``exposed but not infectious'',
``recovered'', ``infected and symptomatic'', or ``infected and
asymptomatic'' to name a few. These models were deterministic in nature
because the transitions between compartments rely on unchanging rates:
the same proportion of each component transitions at each time point and
if you run a deterministic compartmental model (DCM) multiple times you
will alway have the same result.

Stochastic models grew out of this original framework as a way to
capture variability and uncertainty in the systems we wish to study. In
this scenario, some or all transitions between states were based on a
\emph{probability} of transitioning rather than a set rate, meaning that
not the same proportion of a state transitioned at every time step, but
on \emph{average}

Notice the assumptions implicit in the way transitions occur in these
models - it is memoryless, generating an exponential distribution (or
geometric if using discrete time).
\begin{itemize}
\tightlist
\item
  Explain what ERGMs/STERGMs are and why they arose and what are all
  their various strengths, and then discuss how they have been widely
  used in epidemic modeling, for HIV/STIs, but even beyond (Sam has a
  good running list of other applications of EpiModel on the EpiModel
  page).
\item
  Explain how current approaches to STERGMs have required us to make
  simplifying assumptions regarding dissolution, even as they've allowed
  us to do all kinds of awesome things in regards to
  formation/cross-sectional network structure. And the time has come to
  try to improve upon these methods within the STREGM framework. Give
  examples of the various questions that would be improved by doing so
  (e.g.~impacts of EPT in chlamydia)
\item
  And explain that (and perhaps why) it is (currently?) intractable to
  add in general non-memoryless forms of dependence into STERGMs, and to
  estimate these from data, so that it is useful to see how far one can
  get while adding in heterogeneity and complexity but still retaining
  some form of memorylessness.
\end{itemize}
\textbf{Relationship Duration}
\begin{itemize}
\item
  Make sure you dive a bit into some of the ways that people have tried
  to include more complex relationship durations in previous models
  before. In DCMs, this can include having multiple compartments
  represent a single state, because the sum of multiple exponentials is
  not exponential. I can't think of any specific examples where folks
  did this for relationship length (maybe there are none), but certainly
  for other types of transition probabilities.
\item
  And then I'm sure there are papers that have included age-specific
  relational dissolution probabilities in agent-based epidemic models,
  using Martina's definitions of the terms (that is, they model
  relationships, but not with a formal statistical model like ERGMs).
  Try to find a few cases of this to show that it exists. Talk about how
  where their dissolution probabilities come from -- survival
  analysis.\\
\item
  Discuss how survival analysis is the obvious way to consider
  relational dissolution probabilities over time/age, and how a specific
  duration distribution implies a specific survival curve, and vice
  versa.
\item
  dynamic networks require information about relationship duration
\item
  why doing a bit better on dissolution/duration, especially by age,
  will be extra important for thinking about certain interventions for
  certain relatively short-lived infections (e.g.~partner services in
  chlamydia).
\end{itemize}
exponential -- memoryless survival function, exchangeability

\textbf{Age-Related Processes} * additionally, while births and deaths
have been a part of models, only recently are we adding explicit
age-dependent formation terms -- and age changes over the simulation --
what is this effect?\\
* including age in dynamic models may sound straightforward but as we're
going to see adds a surprising amount of complexity * Talk about why
including age-specific rates for anything matters (seems obvious, but
still say it. For example, talk a little about what we know about
age-specific rates of chlamydia or other STIs; or even of infectious
diseases with other transmission types for different reasons related to
contacts. And/or talk about obvious fact that relationship patterns
change in many ways across the life-course. Perhaps summarize a few
modeling papers that bring in age-specific contact patterns in different
ways just to make the general point.

\textbf{Primary Data Source}\\
The empirical behavioral data used in this dissertation are drawn from
the 2006-2010 and 2011-2015 waves of the National Survey of Family
Growth (NSFG). The NSFG surveys men and women aged 15-44 on many aspects
of family life, including but not limited to marriage and divorce,
pregnancy, contraception use, infertility, and other aspects of sexual
and reproductive behavior. In addition to the demographic information
recorded for each respondent and their sampling weights, in this study
we use the data collected in section C of the public use files on each
respondent's recent sexual partnerships with opposite-sex partners in
the last year, with a maximum of three partnerships reported. These data
include the century-month of first sexual contact, the century-month of
last sexual contact, whether the respondent considers this sexual
partnership to be ongoing, and the partnership status (marriage,
cohabitation, or other). We limit the combined data set to those
respondents who report at least one partnership in the last year. Out of
the original 43,303 respondents, our subset contains 32,516 respondents
who report on 40,443 sexual partnerships. Due to the study design, all
relationships that respondents report as ongoing on the day of interview
have right-censored relationship lengths, and there is left-truncation
present due to the large number or relationships that started prior to
the observation window but continued into it.

This dissertation will\ldots{}.First I will use tools from survival
analysis to explore the limitations of the exponential assumption with
regard to the distribution of relationship duration across the
lifecourse. Second, I will document the ways in which age-related ERGM
terms can behave in unexpected ways during certain conditions in
simulations with vital dynamics and other important demographic
processes and explore a variety of potential adjustments.
Lastly\ldots{}.

\chapter{Demography and Dynamic Network Simulations}\label{nets}

In mathematical models the choice of model terms depends on the question
of interest and the underlying patterns in the data and this is no less
true for network models of sexual partnerships developed to understand
disease transmission. Several previously published models using ERGMs
and EpiModel to simulate epidemics focused on men who have sex with men
(MSM) populations in a narrow age range, 18-35 (\emph{cite papers and
also double check that this is true}). These models focused on terms
related to mixing patterns between races, the propensity to form
relationships with individuals relatively close in age, and the
likelihood of concurrent partnerships. Because prevalence of both main
and casual relationships remained relatively stable over the small age
range, the models did not include terms that used age as a predictor of
relationship formation. However, in this project, we focus on
heterosexual relationships over a larger age range (15-45). Unlike the
narrower MSM models, there are large clear differences in the prevalence
of main and casual partnerships over this age range, so we will need to
include terms that involve age in our model (see appendix for breakdown
of empirical data relevant to the model terms). In addition to
influencing the distribution of relationship duration, these differences
are likely to be especially important if we want to use this type of
model to understand the processes that generate the large observed
differences in bacterial STI prevalence by age - one of the broader
goals of this dissertation. However, while individual age is
straightforward to represent in the model, it introduces several
complicating factors for network models largely related to the
boundaries imposed by age range and the aging process. This is not the
first simulation to incorportae vital dynamics, and there are a few
existing corrections relating to these processes already implemented

In this chapter, first I will show the relevant diagnostics
demonstrating that the estimated model itself reproduces key network
statistics in the absence of dynamic vital processes. Then I will show
document how simulations that incorporate individual births, deaths,
aging, and sexual debut lead the network to deviate from these key
statistics. Third, I will explore the possible underlying causes for
these deviations, implement a variety of corrections (some new, some
extensions of previously implmented corrections), evaluate the efficacy
of each, and outline some possible future directions for improved
simulations.

\section{Some Definitions}\label{some-definitions}

\emph{Static Diagnostic:} Simulates networks based on the STERGM model
fit and reports summary statistics for formation model terms. These
networks are not dynamic, thus no dissolution or durational statistics
are reported.\\
\emph{Dynamic Diagnostic:} The dynamic diagnostic simulates the networks
over time and tracks cross-sectional summary statistics for the
formation model terms as well as dissolution model and relational
duration statistics. This is a closed system: nodes do not depart, new
nodes do not arrive, and no nodal attributes change across the
simulation period (i.e.~if you are age 15 when then the diagnostic
begins, you remain 15 for the entire run).\\
\emph{Simulation:} While the static and dynamic diagnostics are run
using simulation methods, when I refer to the ``simulation'' I am
referring to the simulation using the EpiModel API that controls vital
dynamics and other processes like sexual debut and eventually, disease
transmission.\\
\emph{Node / Ego:} An individual in the network.\\
\emph{Nodal Attribute:} A trait of an individual (sex, age, sexual debut
status, etc)

\section{Base Model Overview}\label{base-model-overview}

The base network models we use for this chapter focus largely on the age
effects of relationship formation, with additional terms and structural
offsets. For epidemic modeling purposes it will be important to also
model race/ethnicity due to the existence of large racial dispartiies in
prevalence of several sexually transmitted diseases, but that additional
complexity is not the focus here.

{[}network terms{]}

\section{Closed-System Results}\label{closed-system-results}

One of the several steps to check model fit and convergence is a dynamic
diagnostic. In this diagnostic, we simulate the STERGM for X repetitions
of Y time steps and evaluate the cross-sectional network statistics over
time. At each time step, ties can form and ties can dissolve based on
the model coefficients. If the model is estimated properly and
sufficient MCMC intervals are used, the network formation statistics
should hover around their estimated targets. In this diagnostic we also
evaluate the duration of ties and the rate of tie dissolution to ensure
the dissolution targets are met. It is important to note that this
diagnostic is an indicator of model performance in a closed system: all
nodal attributes are fixed, no nodes exit, and no new nodes enter the
population. In this context the models perform exceptionally well. Not
only is the overall mean degree of each network met, but mean degree by
age in both networks is reproduced well. Additionally, both models reach
the target mean cross-sectional relationship duration after sufficient
time. Any deviation from these targets as we move to the simulation
then, will be related to the introduction of vital dynamics and other
processes like sexual debut that alter nodal attributes that model
terms.
\begin{figure}

{\centering \includegraphics[width=0.6\linewidth]{thesis_files/figure-latex/diagnostic-results-1} 

}

\caption{Comparison: Egodata vs Diagnostic Mean Degree.}\label{fig:diagnostic-results}
\end{figure}
\begin{figure}

{\centering \includegraphics[width=0.6\linewidth]{thesis_files/figure-latex/dynamic-duration-m-1} 

}

\caption{Mean Relationship Lengths in Diagnostic}\label{fig:dynamic-duration-m}
\end{figure}
\section{Overview of Demographic Processes of
Interest}\label{overview-of-demographic-processes-of-interest}

The simulations run using the EpiModel API are distinct from the above
closed-system dynamic diagnostic in that in addition to tie formation
and dissolution at every time step, a series of modules is run that
govern important demographic processes: node departure, node entry,
aging, and sexual debut. Nodes automatically depart the model at age 45.
This boundary was selected for two main reasons: 1) According to the CDC
in their 2018 surveillance report, 97.4\% of all chlamydia infections
were diagnoses in the 15-44 age range (cite surveillance report) and 2)
the National Survey of Family Growth, the empirical data from which we
estimate our model, only surveys adults aged 15-44. There are likely
other sources of information that we could use to increase the age
range, but it did not seem necessary to our questions of interest. Note
that implicit in this decision is the elimination of all reported
relationships among egos aged 15-45 whose \emph{partners} are outside of
this age range. The degree distribution that we actually use to estimate
the model (and are trying to maintain during simulation) looks rather
different than the original distribution shown above, particularly in
the marriage/cohabitation network (see \ref{fig:egodata-2}). We will
consider the consequences of this in a later section.
\begin{figure}

{\centering \includegraphics[width=0.6\linewidth]{thesis_files/figure-latex/egodata-2-1} 

}

\caption{Mean Degree by Ego Age and Relationship Type, Restricted and Unrestricted Alters}\label{fig:egodata-2}
\end{figure}
In addition to the age boundary at 45, all individuals experience the
possibility of dying at each time step. Each node belongs to a class
based on their 5-year-age-category and their sex, and is evaluated for
death at every time step with the probability determined by data from
published in U.S. Vital Statistics documents (cite). Given that our age
range is relatively young, departures due to background mortality are
uncommon relative to the effect of the age boundary on which nodes
depart the model. Nodes enter at age 15 at a rate based on the expected
number of departures per time step in order to keep the population size
relatively stable. Like ASMR, the actual number of entires per time step
is stochastic but maintains a population size within 1-2\% of the
starting size of 50,000 nodes. Each time step in the simulation
represents one week, so nodes age by 1/52 per time step. Nodes enter the
population at age 15 and are evaluated for sexual debut at each time
step, with probability that increases until age 29 to match the
age-at-debut distribution as reported in the NSFG. In accordance with
the data, some individuals will never debut into the heterosexual
population and will therefore never form a tie in these networks.

\section{Initial Simulation Results}\label{initial-simulation-results}

Unlike the dynamic diagnostics, when we run these simulations with
demographic processes, several metrics stray from their target values.
First, the overall network mean degree, or average number of
relationships per person, is too low in both the marriage/cohabitaiton
network and in the casual network (by roughly five and three percent
respectively). These deviations are not incredibly large, but they are
somewhat concerning esepcially becuase Second, the mean degree by age is
severally underpresented in both networks for the youngest ages but
overrepreseted in the mid-30s. Finally, the mean relationship length is
24\% too short in the marriage network but 7\% too long in the casual
network.
\begin{table}

\caption{\label{tab:scen1-tab}Mean Degree and Duration Comparison, Targets and Base Simulation}
\centering
\begin{tabular}[t]{lrrrrrr}
\toprule
  & Mean Degree Target & Base & Pct Off & Mean Duration Target & Base & Pct Off\\
\midrule
Marriage/Cohab & 0.455 & 0.431 & -5.27 & 476 & 365 & -23.32\\
Casual & 0.159 & 0.152 & -4.40 & 95 & 103 & 8.42\\
\bottomrule
\end{tabular}
\end{table}
\begin{figure}

{\centering \includegraphics[width=0.8\linewidth]{thesis_files/figure-latex/scen1-networks-1} 

}

\caption{Base Simulation: Mean Degree by Age.}\label{fig:scen1-networks}
\end{figure}
\section{Boundary Effects \& Existing
Corrections}\label{boundary-effects-existing-corrections}

This section provides an overview of existing corrections for dynamic
networks and dynamic demography and outlines the boundary effect issues
that we will explore in more detail below.

\textbf{1. Formation Approximation} In many cases, a full STERGM cannot
directly be estimated directly due to the computational burden when
networks are large, sparse, and have relatively long relational
durations (which describes many sexual networks). Instead Carnegie,
Krivitsky, Hunter, \& Goodreau (2015) introduce an approximation to full
STERGM estimation that uses the same data: cross-sectional egocentric
network data and information on tie duration. In this approximation, the
static ERGM is estimated using standard techniques. Then the edges
formation term (which represents the base propensity for ties to form
between any two individuals in the network) is decreased by the log odds
of the probability of edge persistence, in effect transforming the
formation term from \emph{prevalence} of ties in the network to the
\emph{incidence} of ties. The explorations below do not attempt to
modify this approach, but instead explore the relationship between the
adjustment of the formation coefficient, the probabilty of tie
persistence as estimated from the long relationship duration expected in
the marriage network relative to the limited observation window per
individual as they arrive and eventually age out.\\
\textbf{2. Right Boundary Existing Correction: Departure} The node
departure correction used in the model estimation-to-simulation workflow
was developed after the observation that when nodes were removed from
the simulation to mimic, for example, background age-specific mortality,
the mean degree of the network became lower than expected, as does the
mean duration of relationships. The logic is relatively straightforward:
the statistical model underlying these network simulations balances the
probability of tie formation with the probability of tie dissolution in
order to maintain a target number of ties in the network. However, when
nodes depart, some additional ties will break due to this process,
lowering mean degree and the mean duration of ties. This node death is
exogenous to the originally estimated statistical model, and therefore
``unexpected''. To counter the lowering of relationship duration (and
subsequently mean degree) related this excess node death, the expected
(endogenous) duration of ties is increased such that the \emph{average}
duration is maintained.\\
The departure correction implemented in previous models has two
components: 1) the mortality rate per time step averaged across the
entire population, and 2) the rate at which individuals depart the
simulation due to the age boundary, calculated as 1/(time steps each
node is expected to be observed in the simluation). In the past this
approach has successfully\ldots{}.

However, due to the large difference in average probability of breaking
a tie upon departer between the marriage/cohab network and the casual
network (due to the distribution of mean degree by age), it seems
unlikely that the departure correction should be the same for both
networks. \textbf{3. Left Boundary Existing Correction: Arrival}
Kritvitsky, Handcock and Morris (2011) - this correction is designed to
maintain the target mean degree of the network in the presence of
changes in the size of the population. In these simulations, while the
exact number of entries at every time step is stochastic, the rate is
designed to keep the population size relatively stable so while we do
use this correction, we will not futher modify it. However, another
arrival-related problem is the issue is that fact that nodes do not
enter the population \emph{with} ties to other nodes. While the model
coefficients suggest a certain average number of relationships among
these young adults, because the nodes age each week, there is a very
limited time frame in which to form relationships and hit these targets.

\section{Considering the effect of older
partners}\label{considering-the-effect-of-older-partners}

Because we observed that in the marriage/cohabitation network there is
an overrepresentation of relationships among the older egos, and the
model coefficients suggests that older nodes are in general more likely
to form relationships than younger nodes (with some tapering as age
increases), we theorized that when a node aged 45 aged out and broke the
tie with their partner (who is likely to be somewhat close in age), that
the partner remaining in the simulation very quickly forms a new
relationship. However, the tie that dissolved as a result of one partner
leaving the simulation due to this age boundary is not a true
dissolution, and these newly formed relationships should not actually
exist because the remaining partner should not actually be eligible to
form a new relationship in the network yet. Perhaps then, these new,
short relationships in older ages contribute both to the lower than
expected mean relationship duration in the marriage network and the
lower than expected mean degree at younger ages.

We consider two ways to address the effect of partners outside the age
boundary. First, we prevent egos whose partners have aged out from
immediately forming new relationships by adding an offset terms for egos
who meet this condition. In this scenario we hope that by preventing new
relationships from forming among egos whose previous relationships were
terminated artificially by the age boundary, the simulation will better
match the data with the restricted alter set and increase the mean
relationship length by generating new relationships at earlier ages. In
the second scenario, we increase the age at which egos depart the
simulation to age 65. While we may not be interested in modeling
individuals older than 45 for epidemiological reasons, it may be
worthwhile to keep them in the simulation over a longer period of time
to avoid the artificial ending of relationships. In this case we hope to
match the empirical mean degree distribution among egos with the
age-unrestricted alter set. However, because we would be simulating
individuals outside the age range in the data we used for estimation, we
may run into additional issues.

\subsection{Offset for Partner
Age-Out}\label{offset-for-partner-age-out}

This scenario adds an offset term to the formation model
(``olderpartner'') for egos whose alters are outside of the 15-45 age
range modeled in the simulation. During estimation there are already
some egos whose partners are outside the age range so they appear to
have degree 0 and do not contributed to the expected edge count but are
flagged as ``olderpartner=1''. During the simulation, if an individual
ages out while they are in a relationship, the remaining partner gets
flagged by the ``olderpartner'' attribute and are prohibited from
forming a new relationship. The probability of becoming available for a
relationship on any future time step is equal to 1/expected duration of
the relationship type, although in the case of the marriage/cohab
network relationships last so long that it's unlikely that a node become
available for the rest of their simulation lifecourse (unless the age
difference between partners was exceptionally large, which is not
impossible).

I don't necessarily expect this to solve the issue of the overall mean
degree, but if we prevent relationships that only exist due to the age
boundary, perhaps these relationships will be distributed among the
younger issues. In turn these relationships would begin earlier, and
possibly increase the average mean duration.

\emph{Results}

The first thing we note is that this offset did not largely influence
the overall mean degree in either network, nor did it increase the mean
relationship duration in the marriage/cohabitation network (mean
relationship length was also unchanged in the casual network, but we did
not necessarily expect it to). However, when comparing mean degree by
age between scenarios, the offset did correct much of the
overrepresentation of relationships at the older ages, and also slightly
increased the mean degree in the youngers (these changes are subtle but
present). The casual network was largely uninfluenced by this offset,
but that would be expected given that older ages are actually less
likely to form casual partnerships than younger ages.
\begin{table}

\caption{\label{tab:scen2-tab}Mean Degree and Duration Comparison, Targets vs Older Partner Offset}
\centering
\begin{tabular}[t]{lrrrrrr}
\toprule
  & Mean Degree Target & Base & Pct Off & Mean Duration Target & Base & Pct Off\\
\midrule
Marriage/Cohab & 0.455 & 0.434 & -4.62 & 476 & 364 & -23.53\\
Casual & 0.159 & 0.152 & -4.40 & 95 & 102 & 7.37\\
\bottomrule
\end{tabular}
\end{table}
\begin{figure}

{\centering \includegraphics[width=0.8\linewidth]{thesis_files/figure-latex/scen2-networks-1} 

}

\caption{Mean Degree Comparison: Base vs Offset.}\label{fig:scen2-networks}
\end{figure}
\subsubsection{Increase Age Boundary}\label{increase-age-boundary}

In this scenario, we hope to move the degree distribution closer to the
egodata distribution with the age-unrestricted alters (blue dots) -- the
distribution that better represents reality. This scenario includes the
offset for ``older partners'' but employs it in a slightly different
fashion. In the previous scenario, edges dissolved artificially when one
of the partners left the model at age 45. We now allow those
relationships to continue as they would normally by increasing the age
of departure in the model to age 65. However, we use the offset to
prevent any nodes older than 45 but not in a relationship from forming
new relationships. Only relationships that began prior both partners
turning 45 exist.

\emph{Results}

First off, it is clear that we can easily recover the marriage and
cohabitations lost to the age boundary among egos in the 35-45 age range
simply by keeping their older partners in the model, even if the data
used to estimate the model did not include these partners. However, this
approach has consequences. Because the model is targeting a mean degree
based on the restricted partner data, the maintenence of relationships
among 35-45 year olds increased the overall mean degree beyond the
target and also comes at the expense of relationships among the younger
ages, the section of the distribution that we already fail to match
well. The mean age of relationships has increased, but this is clearly a
result of the relationship at older ages, thus only a partial success.
The casual network also displays some undesireable qualities similar to
the cohab network. The mean degree is too low at the expense of the
younger age group and the mean relationship length is unchanged.
\begin{table}

\caption{\label{tab:scen3-tab}Mean Degree and Duration Comparison, Targets vs Increased Age Boundary}
\centering
\begin{tabular}[t]{lrrrrrr}
\toprule
  & Mean Degree Target & Sim & Pct Off & Mean Duration Target & Sim & Pct Off\\
\midrule
Marriage/Cohab & 0.455 & 0.472 & 3.74 & 476 & 414 & -13.03\\
Casual & 0.159 & 0.141 & -11.32 & 95 & 104 & 9.47\\
\bottomrule
\end{tabular}
\end{table}
\begin{figure}

{\centering \includegraphics[width=0.8\linewidth]{thesis_files/figure-latex/scen3-networks-1} 

}

\caption{Mean Degree Comparison: Increased Age Boundary.}\label{fig:scen3-networks}
\end{figure}
\emph{Discussion}

conclusion: we keep offset in all future scenarios but not older
partners, older partners may be a good idea in some case but we have to
rethink some things especially if the younger ages are going to perform
worse

\section{Relationship Length \& The Simulation
Window}\label{relationship-length-the-simulation-window}

So far, nothing we have done has largely influenced the issues with
relationship duration in these networks. In the marriage/cohab network,
the mean duration falls nearly 2 years short of the expected length and
in the casual network the mean duration is roughly 3 months too long.
There are a few possible reasons that there may be a mismatch between
the formation and dissolution coefficients in-simulation that may
contribute to these outcomes. Here we explore possible issues related to
the window of observation for each node in the simulation and how that
inluences the observable mean relationship duration.
\begin{figure}

{\centering \includegraphics[width=0.7\linewidth]{thesis_files/figure-latex/plot-expdist-1} 

}

\caption{Predicted Distribution of Relationship Lengths and Simulation Window}\label{fig:plot-expdist}
\end{figure}
{[}will need to re-frame this section if the survival analysis doesn't
happen until chapter 2{]}

From our survival analysis it is clear that the exponential
distribution, even when separated into separate networks by relationship
type, had some serious limitations in its ability to represent the full
distribution of relationship lengths - both by age and across the whole
population. One of the limitations that may pertain to the right tail of
the distribution. An exponential distribution with a mean of roughly 95
weeks (the mean cross-sectional length in the empirical data) has a very
long right tail extending beyond a normal human lifespan, 15 years.
Clearly this tail isn't possible to observe, even less so when you
consider that the window of observation in the simulation is equal to
the age range of the population, 15-45 (30 years).
\ref{fig:plot-expdist} shows the density plot of relationships lengths
that are exponentially distributed with a mean of 95 weeks. While 100\%
of observsations lay within the simulation window, the removal of the
tail lowers the mean observerable relationship length based on this
distribution (the mean of relationship lengths if you remove the
observations that are impossible to occur in the simulation) from 95
weeks to 96 weeks. The mean relationship duration in the casual network
is also shown to demonstrate that the simulation window of each node is
unlikely to contribute to the variation we see in the mean simulated
relationship duration in the same way and as such we will only implment
a correction for the marriage network.

In this scenario, we increase the edges formation coefficient by the
difference between the log odds of the target mean duration and the log
odds of the observable mean duration in the marriage network. If you
recall, this in effect slightly alters the ``edapprox'' approximation
method described above from adjusting the static edges coefficient by
the maximum observable mean duration rather than the rather than the
target mean. We hope that this wil help edges to form on average earlier
in the lifecourse and help us recover missing edges.

While not fully considered here, this issue could also be exacerbated by
the fact that the mean age of marriage/cohab formation is so much later
than the mean age in the casual network. When relationships form earlier
on in the life cycle, the average duration is less curtailed by the age
boundary of the simulation. So in addition to being much shorter than
marriages and cohabitations, and the obverall expected distribution is
less influenced by the obvservation window, the representation of casual
relationships also benefit from their unique distribution.
\begin{table}

\caption{\label{tab:scen4-tab}Mean Degree and Duration Comparison, Targets vs Edapprox Correction}
\centering
\begin{tabular}[t]{lrrrrrr}
\toprule
  & Mean Degree Target & Sim & Pct Off & Mean Duration Target & Sim & Pct Off\\
\midrule
Marriage/Cohab & 0.455 & 0.447 & -1.76 & 476 & 369 & -22.48\\
Casual & 0.159 & 0.152 & -4.40 & 95 & 103 & 8.42\\
\bottomrule
\end{tabular}
\end{table}
\begin{figure}

{\centering \includegraphics[width=0.8\linewidth]{thesis_files/figure-latex/scen4-networks-1} 

}

\caption{Mean Degree Comparison: Edapprox Correction}\label{fig:scen4-networks}
\end{figure}
\textbf{Results}\\
The boost in the edges coefficient sucessfully increased the overall
mean degree of the network to within 2\% of the target mean degree.
However, very little of the increase in mean degree came from the an
increase in the younger ages. Instead, the boost largely only increase
the degree at the peak, which was already over-representing
relationships in the mid-to-late 30s. Additionally, we only gained about
two months in mean relationship duration. This is again likely due to
the increase in mean degree in individuals in their 30s rather than
across the network more evenly.

\section{Departure}\label{departure}

In this scenario, I re-consider the standard implemented departure
correction. As stated earlier, this correction acts to balance out the
``unexpected'' edge dissolutions due to aging out of the simulation or
age-specific mortality by lowering the probability of ``expected'' edge
dissolution and thereby increasing the ``expected'' duration of
relationships. In the past this correction has been the same for both
main (marriage/cohab) and casual partnerships. However, due to the large
difference in the likelihood that a nodal departure causes an edge to
dissolve between the networks, it is not clear that this correction
\emph{should} be same.

The current correction is calculated by adding the likelihood that any
one node departs the network due to aging out multiplied by the mean
weighted age category-specific mortality rate per time step:\\
\[drate = \frac{1}{simulation window} + ASMR_weighted\]

The new correction represents the likelihood that if a node departs, it
also dissolves a edge. \[1-sum(S*D*P)\] where\\
\(S\) is the vector of survival by age category due to aging out\\
\(D\) is the vector of age category specific mortality, and\\
\(P\) is the vector containing the proportional mean degree in each
network by age category

This results in an estimate for the marriahe/cohab network not too far
from the original correction, but a much smaller correction for the
casual network. This makes sense given the very low mean degree at the
oldest ages.
\begin{table}

\caption{\label{tab:departure-tab}Original and Updated Mortality Rates}
\centering
\begin{tabular}[t]{rrr}
\toprule
Original Mortality Rate & Updated Marriage Rate & Updated Casual Rate\\
\midrule
0.0006645 & 0.000749 & 0.0002807\\
\bottomrule
\end{tabular}
\end{table}
\textbf{Results} You should notice that the mean degree of the casual
network was lowered by the implementation of this departure correction.
However this doesn't mean that we're not on the right track. Instead,
look to the mean relationship duration. Because we are no longer
overcorrecting for the number of relationships lost to the age boundary,
the endogenous expectation of relationship duration is not too long and
the cross-sectional mean duration hits the target. The decrease in mean
degree is likely due to these relationships that don't last artifically
long. The marriage network has also experienced some good improvements.
The overall mean degree is now on target, and although the mean
relationship duration is still too low, it has increased by almost half
a year. However, we can see by the distribution of mean degree by age in
both networks that much of the missing relationships are still in the
youngest ages. We will focus on this problem in the next section.
\begin{table}

\caption{\label{tab:scen5-tab}Mean Degree and Duration Comparison, Targets vs Edapprox + Mortality Corrections}
\centering
\begin{tabular}[t]{lrrrrrr}
\toprule
  & Mean Degree Target & Sim & Pct Off & Mean Duration Target & Sim & Pct Off\\
\midrule
Marriage/Cohab & 0.455 & 0.455 & 0.00 & 476 & 387 & -18.7\\
Casual & 0.159 & 0.144 & -9.43 & 95 & 95 & 0.0\\
\bottomrule
\end{tabular}
\end{table}
\begin{figure}

{\centering \includegraphics[width=0.8\linewidth]{thesis_files/figure-latex/scen5-networks-1} 

}

\caption{Mean Degree Comparison: Departure Corrections.}\label{fig:scen5-networks}
\end{figure}
\section{Arrival \& Sexual Debut}\label{arrival-sexual-debut}

The failure of these networks to adequately form relationships among the
youngest ages is yet another form of a boundary problem. The big-picture
problem is that 15-year-olds do not enter the population and bring in
existing relationships. This creates a problem for the model because the
coefficients by age represent the prevalence of relationships at each
those ages, and based on these coefficients and the underlying network
the algorithm makes and breaks a certain amount of relationships at each
time step. It doesn't however, expect to have to form \emph{all} of the
relationships among 15-year olds almost \emph{immediately} up on entry,
which needs to happen 1) to meet the expected mean degree, and 2) in the
presence of aging when there is a very limited window to hit that age
target. This also makes large jumps in the expected mean degree
challenging, like in the marriage network between age 18 and 25 or in
the casual network between ages 15-20. In this section we test two
possible approaches to this problem. The first involves manipulating the
number of individuals eligible for relationships based on the sexual
debut process, and the second takes a more direct approach to manually
calibrate the formation coefficients at certain ages to boost the rate
of formation beyond what the ERGM initially estimated.

I am going to contextualize our scenarios and results in two ways.
First, in the ability to represent the mean degree distribution from the
data (the cross-sectional relationship landscape) and second, in the
accuracy of the in-simulation debut rates (i.e.~who actually forms a
partner at any point in the simulation) - (the longitudinal landscape).
Note that the STERGM methods were developed to maintain certain
cross-sectional statistics so the longtitudnal correctness isn't the
primary goal, but it is important to think about when considering the
broader context.

\subsection{Sexual Debut}\label{sexual-debut}

{[}parts of this section too wordy / possibly not necessary{]}\\
Representing the sexual debut process is both complex and highly
important if we wish to model sexually transmitted diseases in
adolesents and young adults. In the U.S., more than 50\% of all sexually
transmitted bacterial infections such as chlamydia and gonorrhea
diagnosed yearly occur among individuals aged 15-24, but not everyone in
the age group are sexually active, which concentrates the transmissions
into subset of the population and increases the probability of exposure
to an STI for those sexually active moreso than at older ages. It is
important then, to approximate this process in simulation as faithfully
as possible. If too many individuals are able to form sexual
partnerships in the model, we risk under-estimating the risk of exposure
for those sexually active and conversely over-estimating the risk of
exposure if too few are sexually active.

The process by which individuals are labled as ``sexually debuted'' and
thus eligible to form relationships in the model is reasonably
straightforward. Individuls enter the model with a 10.6\% probability of
debut, based on the proportion of 15 year olds in the NSFG who reported
having sexual intercourse with a member for the opposite sex prior to
age 15. For the rest of the age distribtion (15-44), we used the
responses to ``have you ever had sexual intercourse with a member of the
opposite sex'' to generate a cross-section distribution of sexual debut
status. From this data we estimated the yearly probability of debut
among those who had not, and converted that into a weekly probability.
The empirical data and the in-simulation distribution of sexual debut
from the base model scenario are shown in \ref{fig:debut-table}.
\begin{figure}

{\centering \includegraphics[width=0.7\linewidth]{thesis_files/figure-latex/debut-table-1} 

}

\caption{Sexual Debut Status: NSFG vs Simulation}\label{fig:debut-table}
\end{figure}
The major caveat to this approach is that it does not mechanistically
represent the debut process as well as desired . In real life, a person
``debuts'' by entering into a sexual partnership. This puts the model in
a bit of a catch-22 situation: an individual can't debut in real life
until they form a sexual partnership, but in order to form a sexual
partnership in the model, they must already be ``debuted''. We can match
the distribution of sexually debuted individuals, but it is possible
that the number of individiduals who have ``debuted'' does not equal the
number of individuals who have actually formed a relationship at any
point in the simulation.

What if instead we used the ``debut'' term to model the idea of ``sexual
eligibility'' rather than explictly debut? The concept is
straightforward: in most cases, an individual would decide that they are
\emph{ready} to begin having sex some period of time prior to actually
forming a sexual partnership (or transitioning a relationship from a
non-sexual partnership to a sexual one). It is perhaps this underlying
trait that we should model instead, allowing us to model the sexual
history of individuals in our model more completely. Unfortunately our
survey data don't allow us to answer this question directly (i.e.~at
what age did you decide you were ready for sex vs at what age did you
actually start having sex), and the literature has laregly focused more
on individual characteristics and within-partership dynamics that
predict sexual debut rather than quantifying the time to readiness or
the time from readiness to debut (cite that review paper, others).

In this scenario, we change only the probability of having sexually
debuted at entry so that the \emph{effective} debut in simulation
matches the proporion of 15 year olds who report having had sexual
intercourse. We call this the ``eligbility'' scenario. All other
parameters (i.e.~the rate at which debut occurs among the non-debuted)
remain the same. As a reminder, in this model setup, the rate of sexual
debut does not influence the birth/arrival rate in the model. Eligibilty
status however does dictate whether an individual is allowed to form a
relationship, and the number of un-debuted persons was jointly estimated
in the model, so deviations from the original distribution of eligible
individuals will influence the likelihood of tie formation.

\emph{Results}

The switch to an ``eligibility'' metric had some dramatic effects on the
casual network and moderate effects on the marriage/cohabitation
network. In the marriage network, the overall mean degree has increased
to about 7\% greater than the target, and although we do see increases
in the mean degree at younger ages that almost matches the targets, once
again, the majority of the degree increase is seen between ages 30-40.
The increase in the number of relationships that begin at earlier ages
has increased the mean relationship length by roughly one year, but we
still fall far short of the target. In the casual network, the increase
in available egos for casual relationships has led to an incredible
increase both the overall mean degree and in the mean degree in the
under-30 population. The mean relationship duration in this network has
stayed wihtin 1\% of the target, which is good, but wildly
over-represents the total number of relationship at most ages. This
scenario will come the closest to reproducing the actual debut
distbution of the data (\ref{fig:effective-debut-comparison}), but
largely at the expense of the casual network's degree distribution.
\begin{table}

\caption{\label{tab:scen6-tab}Mean Degree and Duration Comparison, Targets vs Expanded Eligibility}
\centering
\begin{tabular}[t]{lrrrrrr}
\toprule
  & Mean Degree Target & Sim & Pct Off & Mean Duration Target & Sim & Pct Off\\
\midrule
Marriage/Cohab & 0.455 & 0.488 & 7.25 & 476 & 401 & -15.76\\
Casual & 0.159 & 0.227 & 42.77 & 95 & 96 & 1.05\\
\bottomrule
\end{tabular}
\end{table}
\begin{figure}

{\centering \includegraphics[width=0.8\linewidth]{thesis_files/figure-latex/scen6-networks-1} 

}

\caption{Mean Degree Comparison: Eligibility.}\label{fig:scen6-networks}
\end{figure}
\subsection{Young Age Formation Boost}\label{young-age-formation-boost}

In this scenario, we change none of the sexual debut parameters in leiu
of adding a nodefactor term for ``young persons'' in each network and
manually modifying 1) the degree of increase in formation, or the
``boost'', and 2) which ages the boost applies to in each network. This
was manually calibrated by trial and error.

There's probably more I can say here.
\begin{table}

\caption{\label{tab:youngboost-tab}Mean Degree and Duration Comparison, Targets vs Young Formation Boost}
\centering
\begin{tabular}[t]{lrrrrrr}
\toprule
  & Mean Degree Target & Sim & Pct Off & Mean Duration Target & Sim & Pct Off\\
\midrule
Marriage/Cohab & 0.455 & 0.497 & 9.23 & 476 & 413 & -13.24\\
Casual & 0.159 & 0.183 & 15.09 & 95 & 96 & 1.05\\
\bottomrule
\end{tabular}
\end{table}
\begin{figure}

{\centering \includegraphics[width=0.8\linewidth]{thesis_files/figure-latex/youngboost-networks-1} 

}

\caption{Mean Degree Comparison: Young Age Formation Boost.}\label{fig:youngboost-networks}
\end{figure}
\begin{figure}

{\centering \includegraphics[width=0.8\linewidth]{thesis_files/figure-latex/effective-debut-comparison-1} 

}

\caption{Percent Debuted In-Sim vs Data, Various Scenarios}\label{fig:effective-debut-comparison}
\end{figure}
\emph{Results}
\begin{itemize}
\tightlist
\item
  yay mean degree marches through age 25 in the marriage network and
  through age 20 in the casual network!\\
\item
  still have problem with over-boosted middle ages and as a result have
  too high overall mean degree\\
\item
  but 413 is the mean relationship duration we expect based on an
  exponential distribution w/ mean 474 with the right tail missing, so I
  actually think we're doing great here\\
\item
  effective debut better than default run but not as good as
  ``eligibility'' aka trade off between cross-sectional correctness
  (mean degree) and longitudinal in some cases. Need to think through
  what this means for the concentration of disease at younger ages
  more.\\
\item
  also effective debut not better than simulation with no debut term or
  process at all LOL
\end{itemize}
\subsection{Summary \& Discussion}\label{summary-discussion}
\begin{table}

\caption{\label{tab:summary-degs}Mean Degree Comparison Summary Table}
\centering
\begin{tabular}[t]{>{\raggedright\arraybackslash}p{1.6cm}>{\raggedleft\arraybackslash}p{1.6cm}>{\raggedleft\arraybackslash}p{1.6cm}>{\raggedleft\arraybackslash}p{1.6cm}>{\raggedleft\arraybackslash}p{1.6cm}>{\raggedleft\arraybackslash}p{1.6cm}>{\raggedleft\arraybackslash}p{1.6cm}>{\raggedleft\arraybackslash}p{1.6cm}>{\raggedleft\arraybackslash}p{1.6cm}}
\toprule
  & Target & Base & Older Partner Offset & Increased Age Boundary & Sim Window Correction & Sim Window + Departure & Increased Eligibility & Young Age Boost\\
\midrule
Marriage/Cohab & 0.455 & 0.431 & 0.434 & 0.472 & 0.447 & 0.455 & 0.488 & 0.497\\
Casual & 0.159 & 0.152 & 0.152 & 0.141 & 0.152 & 0.144 & 0.227 & 0.183\\
\bottomrule
\end{tabular}
\end{table}
\begin{table}

\caption{\label{tab:summary-durs}Mean Relationship Duration Comparison Summary Table}
\centering
\begin{tabular}[t]{>{\raggedright\arraybackslash}p{1.6cm}>{\raggedleft\arraybackslash}p{1.6cm}>{\raggedleft\arraybackslash}p{1.6cm}>{\raggedleft\arraybackslash}p{1.6cm}>{\raggedleft\arraybackslash}p{1.6cm}>{\raggedleft\arraybackslash}p{1.6cm}>{\raggedleft\arraybackslash}p{1.6cm}>{\raggedleft\arraybackslash}p{1.6cm}>{\raggedleft\arraybackslash}p{1.6cm}}
\toprule
  & Target & Base & Older Partner Offset & Increased Age Boundary & Sim Window Correction & Sim Window + Departure & Increased Eligibility & Young Age Boost\\
\midrule
Marriage/Cohab & 476 & 365 & 364 & 414 & 369 & 387 & 401 & 413\\
Casual & 95 & 103 & 102 & 104 & 103 & 95 & 96 & 96\\
\bottomrule
\end{tabular}
\end{table}
\begin{itemize}
\tightlist
\item
  no a one-size-fits-all solution
\item
  increased age boundary has some upsides but need to think more through
  that the actual mean degree target is at that point
\item
  edapprox correction probably not necessary if we're just going to
  boost some coefs later anyway
\item
  may be other ways to calculate the departure correction but I think
  this is on the right track
\item
  still concerned about who in the population is sexually active and the
  under/overestimate of exposure risk
\item
  simulation with no debut term (shown in
  \ref{fig:effective-debut-comparison}) shows that not having a debut
  term has basically the same results for effective debut as the the
  young boost scenario, so it's not necessarily even worth having a
  debut term in the model (yay? oof?)\\
\item
  several of these approachs have clear upsides and downsides but I
  think boosting the coefs, and then negatively boosting coefs at older
  ages might be the best way forward
\end{itemize}
\textbf{future work} * cross-network terms - probably going to do most
analysis on the independent networks but will show both and point at
where there are holes (hey by the time this gets finished maybe Chad
will have already figured this out)\\
* need to think about race and sex differences in formation and
absdiff(age) by sex if we want to use this for applied work

\chapter{Relationship Duration}\label{surv}

As we began to unpack at the end of the previous chapter, there is
reason to believe that the way we stratify relationship duration by
relationship type in the two-network models has some room for
improvement. In this chapter I will first begin with an explanation of
the importance of relationship length on the transmission of STIs,
highlight some key issues related to demographic trends and constraints
of current epidemic models, and set up the scope of analysis. Then I
will use parametric and non-parametric tools from survival analysis to
compare models of relationship duration and some simple extensions to
the exponential (memoryless) framework, with the overall goal of
exploring how well memoryless processes captures the empirical
distributions of relationship length in the National Survey of Family
Growth overall and among various stratifications.

\section{Relationship Length
Overview}\label{relationship-length-overview}

The duration of sexual relationships across a population is a key
component of the network structure responsible for either exposing
individuals to or protecting individuals from sexually transmitted
infections (STIs). Relationship duration determines the length of
exposure to pathogens, or in the case of a disease-free monogamous
partnership, protection from pathogens. In addition to dictating this
period of possible exposure, relationship durations relative to the
pathogen-specific duration of infection are an important driver of how
quickly STIs can spread throughout a population. Transmission beyond a
pair of actors for infections with short durations relative to
relationship lengths is challenging and slow, and it is more likely that
an infection will be detected and treated or resolved naturally prior to
the dissolution of the relationship. If the duration of infection and
duration of relationships are more equal, there is a greater chance that
the infection can spread to future partners and throughout the network.
When partnerships overlap, transmission pathways increase even among
those individuals with few lifetime partners, and this effect is even
greater when the duration of overlap is large (Armbruster, Wang, \&
Morris, 2017, Morris \& Kretzschmar (1997)).

The pattern of relationship durations across the life-course is also
important because STIs often have distinct age patterns in terms of
prevalence. Individual age is often used as a predictor for risky sexual
behavior, but there is additional complexity when considering the effect
of age on the duration of relationships across the life-course. Young
age likely influences the immediate intentions for relationships
(i.e.~serious or casual), and the frequency at which individuals form
new relationships, but somewhat paradoxically it is also true that the
only people who can report extremely long relationships are those who
started them at young ages. This also introduces complex sampling issues
because most data on relationship durations is collected
cross-sectionally or retrospectively -- not longitudinally (see
description of methods below for more on this). Given the importance of
relationship duration to features of STI epidemiology discussed above,
there is growing interest in improving the representation of relational
durations in dynamic network models used to study epidemics.

As we used in the first chapter and will continue to use throughout this
dissertation, one common class of models used to understand network
influences on patterns of STI transmission is known as separable
temporal exponential-family random graph models (STERGMs). These models
are governed by two expressions: one that represents the set of
processes that influence the formation of relationships, and a
comparable one for dissolution (Krivitsky \& Handcock, 2014). We have
previous explored some corrections to these expressions related to
unexpected effects of certain demographic processes, but here we explore
assumptions inherent in the dissolution component in more detail. The
current standard practice for the dissolution models in this modeling
framework assumes that once a relationship begins, its persistence is
governed by a constant hazard. This memoryless process is a convenient
simplifying assumption, adopted because most hypotheses being explored
relate to processes impacting network formation or cross-sectional
structure. However, it is unlikely that this assumption faithfully
represents the distribution all relationship durations we observe across
a wide range of ages.

Several recent models have begun to address this simplification by
splitting out relationships into two categories: the first, marriages
and cohabitations or main partnerships, and the second, persistent or
casual partnerships. These are then modeled as separate networks
simultaneously. This strategy is what we employed in the first chapter.
By structuring the model in this fashion, each network has a hazard of
dissolution specific to its type. (These models often have a third
network for one-time sexual contacts which last only one time-step, but
this network is not the focus of our study). While these models are
indeed able to reproduce the mean relationship lengths drawn from
empirical data, it remains unknown how well these strategies reproduce
the full distribution of lengths observed. In particular, the memoryless
assumption means that the modal length of main partnerships remains near
zero across all ages, which basic intuition says is not true and
descriptive data analysis confirms. Other work has considered
disaggregating relational durations by a single demographic attribute of
their members related to a hypothesis or prevention modality being
explored, but again with no further effort to capture the full
distribution, particularly by age (Goodreau et al., 2017,
({\textbf{???}})).

\section{Data}\label{data}

The combined 2006-2015 waves of the National Survey of Family Growth
once again provide the empirical data for this investigation although
this time we use all of the data on relationships and relationship
duration (except for one-time partners) rather than only using the
active cross-section of relationships. As mentioned briefly in the
introduction, the survey design makes the information on relationship
duration somewhat more complex to analyze than the other questions of
interest. Each participant, if they have indicated they have had sexual
intercourse, is asked about their three most recent relationships that
have either started or ended within the last year. We then define
relationship duration as the difference between the month the ego
reported first having sex with this partner and either the last month
they reported sexual intercourse with that partner or the day of the
interview if the relationship is ongoing. All relationships active on
the day of the interview have right-censored duration since we do not
know if/when they end. Additionally, because we calculate duration from
retrospective information, we also introduce left truncation that biases
mean duration estimates upwards. For example, if someone reports having
one monogamous 15-year relationship, we essentially know their 15-year
relationship history. However, if someone has serial short relationships
or long time intervals between relationships, we do not see these
relationships going back 15 years, so we actually gather different
amounts of information from each participant. Figure \ref{fig:censoring}
highlights these phenomena. Blue relationships are those lengths that we
observe via the NSFG questionnaire. Red extensions to the blue lines
represent theoretical true durations among the right-censored
relationships. Green lines are those hypothetical relationships that
could have occurred in the intervals that we do not observe for each
participant. Many methods in survival analysis have corrections for
these types of censoring and are employed in the relevant analyses.
\begin{figure}

{\centering \includegraphics{thesis_files/figure-latex/censoring-1} 

}

\caption{Known and Censored Relationships in NSFG}\label{fig:censoring}
\end{figure}
\section{Methods}\label{methods}

First, the relational duration data is displayed using histograms
(overall, by relationship type, and by age category). These histograms
are not corrected for any censoring, therefore are solely used to get a
visual sense of the underlying patterns. In the main analysis we explore
several parametric survival models to gain insights into the underlying
heterogeneity in hazard of dissolution. The goal here is not to find the
most perfect fitting model, but to explore some simple extensions of the
exponential that may be implemented within the constraints of current
epidemic network models to better capture the full distribution of
relationship lengths. Unless otherwise specified, the parametric models
are fit using the R package `flexsurv' adjusting for the right-censoring
and left-truncation ({\textbf{???}}). All models use the survey weights
provided by the NSFG, which weight the observations to the age, sex, and
race composition of the United States. Model fit is evaluated by the
Akaike Information Criterion (AIC) and visually by using a Modified
Kaplan-Meier (following Burington et al. (2010) and fit using the R
package `survival') as reference curves to compare the survival of the
empirical relationships to that of the fitted models. Most extensions to
the exponential will be fit twice: once using the all relationships and
once stratified by relationship type to reflect the current standard
practice in the literature.

(All parameters in these fitted models are statistically significant (p
\textless{} 0.001). maybe put that on the graphs. although significance
here doesn't mean we gain that much model fit most of the time)

\section{Descriptive Histograms}\label{descriptive-histograms}

At first glance, the histogram of all relationships looks like something
we would expect from an exponential distribution: a high decay right at
the beginning, and a long right tail. However, it is clear from
\ref{fig:hist-reltype} that this shape is primarily driven by the casual
relationships rather than the marriages and cohabitations. The marriages
and cohabitations are not uniformly distributed, but have a slower,
linear-looking decay without an initial steep peak. These trends are
largely maintained if we break these types down further by age category
of the reporting ego, although interestingly the marriage and
cohabitations do look increasingly uniform with age. The casual
relationships all look relatively exponential, but of note is that the
immediate drop in survival is of a lesser relative magnitude with
increasing age. While not shown here, the histograms of solely the
active relationships on the day of the interview look remarkable similar
but have slightly fewer very short relationships as we would expect.
Right off the bat we have indications that for certain relationship
types, and for certain age groups, a simple constant hazard of
dissolution may not accurately capture the distributed overall and over
the lifecourse.
\begin{figure}

{\centering \includegraphics[width=0.7\linewidth]{thesis_files/figure-latex/hist-all-1} 

}

\caption{All Relationships either Current or Ended in the Last Year}\label{fig:hist-all}
\end{figure}
\begin{figure}

{\centering \includegraphics[width=0.8\linewidth]{thesis_files/figure-latex/hist-reltype-1} 

}

\caption{All Relationships either Current or Ended in the Last Year, By Type}\label{fig:hist-reltype}
\end{figure}
hists by age cat needs to be formatting differently, latex not expecting
such a large plot, working on it\ldots{}.
\begin{figure}

{\centering \includegraphics{thesis_files/figure-latex/hist-agecat-1} 

}

\caption{All Relationships either Current or Ended in the Last Year, by Age Category and Type}\label{fig:hist-agecat}
\end{figure}
\section{Duration-Only Models}\label{duration-only-models}

As a first pass, we fit several duration-only (covariate free) models
using 3 different but related distributions: the exponenential, the
weibull, and the gamma. We chose the last two because they are related
to the exponential and a better fit using these distributions would
represent that there is a heterogeneity in the data that the
single-parameter exponential doesn't capture. No STERGM developed for
epidemic models has used either of these distributions in a dissolution
model, but it is not impossible.

{[}need to re-do graph, make it prettier{]}

As we expect, a single exponential does not capture the full
distribution of relationships well. All of the distributions fail to
capture the very long, almost flat right tail of the data but the
exponential seems to capture these relationships the worst. The
exponential also fails to capture the the rapid decay at the start of
the distribution - the shortest relationships. Both the gamma and
weibull perform better than the exponential, but the weibull captures
the short relationships somewhat better. In line with our expectations
based on histograms, there is clear important heterogeneity in the data.
However, all three fail to capture the few remaining long relationships
that are not expected to end in the roughly 30 years of observation time
shown on the x-axis.
\begin{figure}

{\centering \includegraphics{thesis_files/figure-latex/exp-dist-surv-1} 

}

\caption{Various Duration-Only Survival Models, All Relationships}\label{fig:exp-dist-surv}
\end{figure}
\section{Exponential and Relationship
Type}\label{exponential-and-relationship-type}

Here we re-plot the exponential model fit on its own without the others
for reference (\ref{fig:exp-network}). \ref{fig:exp-networktype}
stratifies the relationships into the current practice in the
literature: into those labeled by the participants as marriages and
cohabitations, and into all other relationships (we often think of these
as ``casual'' or ``persistent'' partnerships - all non-marriage/cohabs
that last for more than one sexual encounter). Each relationship type
then has its own hazard of dissolution, but within each relationship
type the hazard is constant. The curve for the casual partnerships fits
remarkably well. The curve for marriage/cohabitations however, does not.
This model over-represents the survival of relationships that last less
than 200 weeks (\textasciitilde{} four years), but under-estimates the
survival of longer relationships. Clearly there is more heterogeneity
here that we will try to tease out in the next examples.

{[}K-S test for the casual model fit? i.e.~are they plausibly two
samples from the same underlying distribution?{]}
\begin{figure}
\includegraphics[width=0.8\linewidth]{thesis_files/figure-latex/exp-network-1} \caption{Kaplan-Meier vs Exponential - All Relationships}\label{fig:exp-network}
\end{figure}
\begin{figure}
\includegraphics[width=0.8\linewidth]{thesis_files/figure-latex/exp-networktype-1} \caption{Kaplan-Meier vs Exponential - By Relationship Type}\label{fig:exp-networktype}
\end{figure}
\section{Simple Extensions}\label{simple-extensions}

Here we explore add a variety of covariates to the exponential models to
hopefully tease out the remaining heterogeneity while using covariates
that often also are relevant to STERGM formation models (age category
and race), and additionally explore splitting out the
marriage/cohabitations into two groups to be estimated separately.

\subsection{Age Category}\label{age-category}

There are only very small differences in the expected survival of casual
relationships by current age category, and we gain very little from
splitting out each curve by age group (see AIC table) - it seems that
casual relationships that do not or haven't yet moved into a
cohabitation or marriage fail at a fairly similar rate regardless of the
current age of the reporting ego. For the marriage/cohabs, the current
age category is better representative for the youngest ages, the 15-19
year olds and the 20-24 year olds, but less so for the older age groups.
We expect this pattern based on the increasingly uniform distribution in
the histograms with increasing age. This is perhaps not surprising, in
that the age relationship lengths is at least partly an emergent
property rather than a causal one. That is, no individual can have a
relationship that has lasted longer than they have been sexually active,
so the range of relationship lengths for young age categories is
relatively small. Meanwhile, the older age categories are challenging to
represent because the possible range of relationships is so much larger,
and are likely influenced not only by dissolution probabilities but also
by the changing formation probabilities over the life-course -- that is,
older people in long-term relationships do not start new relationships
at the same rate as others, and thus have relatively few relationships
that are short. The poor fit at older ages in the overall model (with
combined )

in appendix we show not current age but other models fit to age at
relationship initiation and similar effects
\begin{figure}

{\centering \includegraphics[width=0.7\linewidth]{thesis_files/figure-latex/agecat-1} 

}

\caption{Kaplan-Meier vs. Constant Hazard by Current Age Category}\label{fig:agecat}
\end{figure}
\begin{figure}

{\centering \includegraphics[width=0.7\linewidth]{thesis_files/figure-latex/agecat-reltype-1} 

}

\caption{Kaplan-Meier vs. Constant Hazard by Current Age Category and Rel Type}\label{fig:agecat-reltype-1}
\end{figure}\begin{figure}
{\centering \includegraphics[width=0.7\linewidth]{thesis_files/figure-latex/agecat-reltype-2} 

}

\caption{Kaplan-Meier vs. Constant Hazard by Current Age Category and Rel Type}\label{fig:agecat-reltype-2}
\end{figure}
\subsection{Race}\label{race}

Here we add a covariate for race/ethnicity of reporting ego. For the
casual relationships, like we saw for age category, there are such small
differenes between groups that this covariate adds little to the
picture. The marriage/cohabs relationships tell a slightly different
story. The Non-Hispanic Black population has a clear separation from the
Hispanics, Non-Hispanic Whites, and all others. Among Blacks the
exponential model follows the curve of the corresponding Kaplan-Meier
much closer than all other races which tend to repeat the overall
pattern we saw in the duration-only model: a large overestimation of the
survival of relationships less than 4 years in length, and then
under-estimating the survival in the much flatter right tail.
\begin{figure}

{\centering \includegraphics[width=0.7\linewidth]{thesis_files/figure-latex/race-reltype-1} 

}

\caption{Kaplan-Meier vs. Constant Hazard by Current Age Category among Casual Relationships}\label{fig:race-reltype}
\end{figure}
\begin{figure}

{\centering \includegraphics[width=0.7\linewidth]{thesis_files/figure-latex/race-reltype2-1} 

}

\caption{Kaplan-Meier vs. Constant Hazard by Current Age Category among Marriage/Cohabs}\label{fig:race-reltype2}
\end{figure}
\subsection{Three Relationship Types}\label{three-relationship-types}

The next two models test how appropriate it is to group relationships
defined as marriages and cohabitations into the same dissolution model,
as has been done in recent STERGMs. We see clear evidence that marriages
and cohabitations have distinct hazards of dissolution, and that within
these three types the exponential captures the distribution well. These
results are similar to other work in family demography that has shown
significant differences in the risk of dissolution between cohabitations
and marriages due to variation in joint lifestyles (van Houdt and
Poortman 2018). These results suggest that cohabitation represents a
distinctly separate type of relationship from marriages and other casual
relationships and we could improve the overall accurary of our
dissolution models if we captured this additional heterogeneity in
risk.\\
{[}do K-S test here?{]}
\begin{figure}

{\centering \includegraphics[width=0.7\linewidth]{thesis_files/figure-latex/threerels-1} 

}

\caption{Kaplan-Meier vs. Constant Hazard by 3 Rel Types}\label{fig:threerels}
\end{figure}
\section{Summary of Model Fits and
Takeaways}\label{summary-of-model-fits-and-takeaways}
\begin{itemize}
\item
  {[}need to make table of AIC and comments about visual fit{]}
\item
  takeaway here is most models that rely on common stratifications
  (casual vs marriage/cohab, age and race) don't fit well but the
  3-relationship exponential does
\item
  neither current age category or race of reporting ego make a
  substantial difference in the dissolution hazard for relationships
  that never or haven't yet transitioned to a cohabitation or marriage.
  it seems that a simple exponential for this relationship type is a
  reasonable approximation.
\item
  there are differences in age category and race for marriage and cohabs
  but the age effects are likely more emergent than casual, and the race
  effect, particularly for non-hispanic blacks, is likely due to the
  higher prevalence of cohabiting relative to marriages compared to the
  other race groups (maybe should have a table to show that in NSFG.)
  Outside of non-hispanic blacks however, an exponential is not a
  reasonable approximation to the K-M within the other groups.
\end{itemize}
\section{Young Population}\label{young-population}

In this section we focus on the fit of models when only participants
aged 15-29 are considered. We know from the above work that the most
challenging age groups to fit are the oldest given their wide
distribution of relationship length, and depending on the research
question of interest it may not be as important to model these
relationships perfectly. Because the majority of chlamydia diagnoses
occur in under-30 year olds, in the next chapter we could prioritize
capturing that distribution rather than that of the full population.
\begin{figure}

{\centering \includegraphics[width=0.8\linewidth]{thesis_files/figure-latex/young-exp-1} 

}

\caption{Kaplan-Meier vs Exponential - By Relationship Type, 15-29}\label{fig:young-exp}
\end{figure}
\begin{figure}

{\centering \includegraphics[width=0.8\linewidth]{thesis_files/figure-latex/young-network2-1} 

}

\caption{Kaplan-Meier vs Exponential - By Relationship Type, 15-29}\label{fig:young-network2}
\end{figure}
\section{Note on target relationship duration
estimation}\label{note-on-target-relationship-duration-estimation}

For comparability with the published literature, the network models
estimated in the first chapter calculated the target mean relationship
length by taking the mean length of all active relationships in each
relationship category (marriage/cohab and casual). Previous work has
shown that if the empirical data follows an exponential distribution,
the left-truncation and right-censoring present in the data due to the
data-generating process (cross-sectional survey with 1-year
retrospective questions) cancel each other out (reference kirk diss? or
was that pavel). Thus, the mean of relationship lengths active on the
day of the interview is an accurate estimate of the true mean duration.
\begin{itemize}
\item
  we also show here the distribution of cross-sectional relationship
  lengths in the ``Young Age Formation Boost'' Models to the empirical
  data and parametric models
\item
  differences in mean if you fit an exponential model accounting for
  left-truncation and right censoring (mean based on estimated model for
  casual rels is much shorter, and marriage/cohabs much much longer)
\item
  that is part of the reason the curves from the ``young boost'' model
  from chapter 1 look so different. - although interestingly the
  exponential marriage/cohab fits so poorly in the first 4 years that
  the dissolution hazard based on the cross-sectional mean actually
  looks closer to the kaplan-meier
\item
  but also not that if we move towards a transitional model these aren't
  actually the targets we want, so this may be a moot point. but I will
  probably use estimates from the fitted models as the mean duration
  targets in 3rd chapter
\end{itemize}
\begin{figure}
\includegraphics[width=0.8\linewidth]{thesis_files/figure-latex/exp-network2-1} \caption{Kaplan-Meier vs Exponential - By Relationship Type}\label{fig:exp-network2}
\end{figure}
\section{Mixture Models}\label{mixture-models}

appendix? put immune fraction model in there too?\\
these models are kind of cool in a ``look what we can fit'' kind of way
but aren't really helpful in terms of finding a simple solution for
epidemic models

all models with latent components will be personally developed and
models will be fit using the maxLik package (Jackson 2016; Henningsen
and Toomet 2011)(, Henningsen \& Toomet, 2011).

\section{Discussion}\label{discussion}
\begin{itemize}
\tightlist
\item
  when do models based on exponential do ok and when don't they
\item
  options for future epidemic models
\item
  current setup - is getting marriage network as important as getting
  the casual network? probably ok for younger pop
\item
  3 networks (casual, cohab, marriage) (seems onerous and given the
  age/race differences this would be a strange setup overall)
\item
  transitional network
  \begin{itemize}
  \tightlist
  \item
    valued tergms don't exist yet
  \item
    could be done via epimodel but would likely require great deal of
    calibration
  \end{itemize}
\end{itemize}
\chapter{Chlamydia, Acquired Immunity, \& Expedited Partner
Therapy?}\label{ept}

copying over some text from diss proposal:

C. trachomatis is an obligate intracellular bacterium transmitted
through sexual contact among humans. Chlamydial infections are most
often asymptomatic. Untreated infections in women are an additional
public health concern because they can lead to a variety of sequalae
including pelvic inflammatory disease, scarring of ovaries and fallopian
tubes, ectopic pregnancies, chronic pain, and infertility. Repeat
infections are common and are an additional risk factor for the
development of the above sequelae (Brunham and Rey-Ladino 2005). There
is a great deal of uncertainty regarding the natural history of
chlamydia, but the duration of infection for untreated individuals is
generally thought to be up to 6 months for men and a year or more for
women (Golden et al. 2000; Satterwhite et al. 2013). Chlamydia is
usually treated with azithromycin or doxycycline, and unlike other
common STIs like syphilis and gonorrhea, true antibiotic resistance is
rare (Kong et al. 2015).

Chlamydia is the most common reportable disease in the United States and
incidence, particularly adolescents and young adults aged 15-29, is
increasing nationwide. The Centers for Disease Control and Prevention
(CDC) estimates that half of all new STI infections (including
gonorrhea, syphilis, and others) occur in those aged 15-24 despite them
making up only a quarter of the sexually active population. Even in
places like King County, Washington, where overall rates have remained
stable, longstanding acknowledged disparities in prevalence by race are
marked and continue to increase (2015 SKCPH STD Report). These rates are
particularly distressing in light of the fertility consequences of
long-term infection and reinfection: it is estimated that in King
County, over 60\% of non-Hispanic Black women have had at least one
chlamydia infection by age 34 (a rate 5x higher than non-Hispanic White
women) and 1 in 500 of non-Hispanic Black women develops
chlamydia-associated tubal factor infertility over their life-course
(Chambers et al. 2018).

The United States has some of the highest STD rates in the
industrialized world, and despite this, funding for public health
programs dedicated to these issues has largely declined (CDC 2016 STD
Report). As a result, few health departments are able to offer
traditional partner notification services, where a patient who tests
positive for an STI gives the contact information of their recent sex
partners to the health department, and the department then contacts
their partners with the hope that these partners will then get tested
and, if necessary, treated. Expedited partner therapy (EPT) was
developed with this scenario in mind (See figure 2). Under EPT, a
patient who tests positive, upon receipt of their own treatment,
receives either additional antibiotic pills for their recent sexual
partners or prescriptions for treatment that their partners can fill.
The patient then is expected to hand-deliver either the treatment or
prescription to their partner(s), who take the medicine at their own
discretion and without the need for a positive lab test. By using these
actors to essentially leverage their sexual network in reverse, this
system hopes to decrease the time to treatment for all possible infected
partners and increase the total number of partners treated. It can also
reduce re-infection among the index patients if the partnerships are
ongoing. There have been several clinical trials of EPT across the US
(and Europe), including Washington State. These trails demonstrated that
relative to traditional referral practices, EPT provision increased the
proportion of partners who were ultimately treated, reduced the number
of individuals who were re-infected at follow-up, and was less costly if
at least 30\% of partners were treated via EPT (CITE). Despite these
results and a growing body of evidence in support, widespread
implementation of EPT has been slow and there are still many questions
to be answered.

also -- EPT as a tool for health equity, not just effective
population-level decrease in prevalence -- can be more effective in
high-prevalance groups?

Annals of Internal Medicine Article High Incidence of New Sexually
Transmitted Infections in the Year following a Sexually Transmitted
Infection: A Case for Rescreening - Peterman et al

Arrested Immunity Hypothesis One of the paradoxes in era of modern
public health is that chlamydia incidence has actually increased overall
in the presence of mass control programs. In Sweden, Norway, Finland and
Canada the rates initially decreased but then resumed increasing, and in
Australia, United States, and the United Kingdom the rates never stopped
increasing even after program initiation, although this second pattern
has been attributed to the challenges of implementing control programs
consistently throughout a large population (Brunham and Rekart 2008).
These areas now experience incidence rates higher than rates prior to
introduction of control programs. Additionally, a regression analysis
using data from family planning clinics in Region X of the United States
(Alaska, Washington, Idaho, and Oregon) found that, after controlling
for any changes in demographics, sexual behaviors, and increased
sensitivity of clinical tests, there was a remaining 5\% `true' and
unexplained annual increase in chlamydia positivity from 1997-2004 (Fine
et al. 2008). In response to these and other examples of unabated
chlamydia infection in the presence of control programs, Brunham and
Reckart have proposed the arrested immunity hypothesis (Brunham and
Rekart 2008). Under this hypothesis, early detection and treatment of
chlamydia interrupts the development of acquired immunity, making
treated individuals particularly vulnerable to reinfection almost
immediately after treatment. While we have no natural history studies of
chlamydia infection in humans that address the development, duration,
and extent of immunity, there is growing evidence beyond rodent models
and trends in incidence that partial immunity can develop and play a
role. Rodent models of chlamydial infection suggest that a high
proportion are able to resolve their primary infection and are
temporarily resistant to infection. Rodents that then eventually become
reinfected with chlamydia have a shorter duration of disease, lower
pathogen load and decreased inflammatory response (Rank et al. 2003).
However, it has also been shown that treatment early in the course of
infection interrupts the development of this protective immunity (Su et
al. 2002). There is also some indirect evidence in humans. A 2010 review
article acknowledged that in several studies of infection status among
couples, the rates of discordance (i.e.~one partner is infected while
the other is not), are higher for chlamydia than for gonorrhea and that
this discordance increases with age, providing indirect evidence for
some level of protective immunity to chlamydia that increases with age,
likely due to exposure over time. There is little immunity that develops
to gonorrheal infection due to high levels of antigenic variation
(Batteiger et al. 2010). Recent modeling using data from both the UK and
United States has demonstrated that at least some immunity to chlamydia
following natural clearance is necessary to generate observed patterns
in incidence (Omori, Chemaitelly, and Abu-Raddad 2019). These questions
are particularly relevant in the context of expedited partner therapy,
where the goal is to interrupt transmission by treated individuals and
their partners as quickly as possible. However, due to the arrested
immunity of those treated quickly, if the timing of delivery and uptake
of partners is not sufficient, the initially treated is likely at higher
risk of reinfection than under the standard referral scenario. If
sufficient numbers of partners are treated effectively and quickly and
transmission throughout the network is greatly diminished, then EPT may
be able to overcome the effects of this arrested immunity.

\chapter*{Conclusion}\label{conclusion}
\addcontentsline{toc}{chapter}{Conclusion}

We conclude.

\appendix

\chapter{The First Appendix}\label{the-first-appendix}

\section{Data and Model Terms}\label{data-and-model-terms}
\begin{figure}

{\centering \includegraphics[width=0.6\linewidth]{thesis_files/figure-latex/egodata-prep-1} 

}

\caption{Mean Degree by Ego Age and Relationship Type.}\label{fig:egodata-prep}
\end{figure}
include plots for the other terms - absdiff(sqrtage), concurrent,
debuted etc

several general trends in relationship formation (finish write-up and
cite) --
\begin{itemize}
\tightlist
\item
  individuals often select partners that are not their exact age
\item
  this difference in partner ages often increases over the life course
  (i.e.~adults usually have wider age differences between their partners
  than do adolescents)
\item
  it is common for men to partner with younger women (although the sex
  differences in relationship formation are not explored in this model,
  it's important to note that in a more realistic model the effect of
  aging out would disproportionately affect the women whose partners age
  out before them
\end{itemize}
(include model terms and coefs and explain terms) (full description of
EpiModelHIV module flow w/ parameters in appendix, brief overview here)

Cross network terms - we're going to avoid them due to complications

\chapter{The Second Appendix}\label{the-second-appendix}

more technical stuff in here?

\chapter*{Colophon}\label{colophon}
\addcontentsline{toc}{chapter}{Colophon}

This document is set in \href{https://github.com/georgd/EB-Garamond}{EB
Garamond}, \href{https://github.com/adobe-fonts/source-code-pro/}{Source
Code Pro} and \href{http://www.latofonts.com/lato-free-fonts/}{Lato}.
The body text is set at 11pt with \(\familydefault\).

It was written in R Markdown and \(\LaTeX\), and rendered into PDF using
\href{https://github.com/benmarwick/huskydown}{huskydown} and
\href{https://github.com/rstudio/bookdown}{bookdown}.

This document was typeset using the XeTeX typesetting system, and the
\href{http://staff.washington.edu/fox/tex/}{University of Washington
Thesis class} class created by Jim Fox. Under the hood, the
\href{https://github.com/UWIT-IAM/UWThesis}{University of Washington
Thesis LaTeX template} is used to ensure that documents conform
precisely to submission standards. Other elements of the document
formatting source code have been taken from the
\href{https://github.com/stevenpollack/ucbthesis}{Latex, Knitr, and
RMarkdown templates for UC Berkeley's graduate thesis}, and
\href{https://github.com/suchow/Dissertate}{Dissertate: a LaTeX
dissertation template to support the production and typesetting of a PhD
dissertation at Harvard, Princeton, and NYU}

The source files for this thesis, along with all the data files, have
been organised into an R package, xxx, which is available at
\url{https://github.com/xxx/xxx}. A hard copy of the thesis can be found
in the University of Washington library.

This version of the thesis was generated on 2020-11-12 12:05:22. The
repository is currently at this commit:

The computational environment that was used to generate this version is
as follows:
\begin{verbatim}
- Session info ---------------------------------------------------------------
 setting  value                       
 version  R version 3.6.1 (2019-07-05)
 os       macOS Catalina 10.15.3      
 system   x86_64, darwin15.6.0        
 ui       X11                         
 language (EN)                        
 collate  en_US.UTF-8                 
 ctype    en_US.UTF-8                 
 tz       America/Los_Angeles         
 date     2020-11-12                  

- Packages -------------------------------------------------------------------
 package        * version    date       lib
 ape              5.3        2019-03-17 [1]
 assertthat       0.2.1      2019-03-21 [1]
 backports        1.1.9      2020-08-24 [1]
 bookdown         0.20.2     2020-08-06 [1]
 broom            0.5.2      2019-04-07 [1]
 callr            3.4.3      2020-03-28 [1]
 cellranger       1.1.0      2016-07-27 [1]
 cli              2.0.2      2020-02-28 [1]
 coda             0.19-3     2019-07-05 [1]
 codetools        0.2-16     2018-12-24 [1]
 colorspace       1.4-1      2019-03-18 [1]
 crayon           1.3.4      2017-09-16 [1]
 data.table       1.12.8     2019-12-09 [1]
 DBI              1.1.0      2019-12-15 [1]
 ddaf           * 0.0.0.9000 2020-10-07 [1]
 DEoptimR         1.0-8      2016-11-19 [1]
 desc             1.2.0      2018-05-01 [1]
 deSolve        * 1.27.1     2020-01-02 [1]
 devtools       * 2.3.1      2020-07-21 [1]
 digest           0.6.25     2020-02-23 [1]
 doParallel       1.0.15     2019-08-02 [1]
 dplyr          * 1.0.2      2020-08-18 [1]
 ellipsis         0.3.1      2020-05-15 [1]
 EpiModel       * 1.7.5      2020-01-07 [1]
 ergm           * 3.10.4     2019-06-10 [1]
 evaluate         0.14       2019-05-28 [1]
 fansi            0.4.1      2020-01-08 [1]
 farver           2.0.3      2020-01-16 [1]
 flexsurv         1.1.1      2019-03-18 [1]
 forcats        * 0.4.0      2019-02-17 [1]
 foreach          1.4.7      2019-07-27 [1]
 fs               1.5.0      2020-07-31 [1]
 generics         0.0.2      2018-11-29 [1]
 ggfortify      * 0.4.7      2019-05-26 [1]
 ggplot2        * 3.3.2      2020-06-19 [1]
 ggpubr         * 0.2.2      2019-08-07 [1]
 ggsignif         0.6.0      2019-08-08 [1]
 ggthemes       * 4.2.0      2019-05-13 [1]
 git2r            0.27.1     2020-05-03 [1]
 glue             1.4.1      2020-05-13 [1]
 gridExtra      * 2.3        2017-09-09 [1]
 gtable           0.3.0      2019-03-25 [1]
 haven            2.1.1      2019-07-04 [1]
 here           * 0.1        2017-05-28 [1]
 hms              0.5.0      2019-07-09 [1]
 htmltools        0.5.0      2020-06-16 [1]
 httr             1.4.2      2020-07-20 [1]
 huskydown      * 0.0.5      2020-08-06 [1]
 iterators        1.0.12     2019-07-26 [1]
 jsonlite         1.7.0      2020-06-25 [1]
 kableExtra     * 1.1.0      2019-03-16 [1]
 km.ci            0.5-2      2009-08-30 [1]
 KMsurv           0.1-5      2012-12-03 [1]
 knitr            1.29       2020-06-23 [1]
 labeling         0.3        2014-08-23 [1]
 lattice          0.20-38    2018-11-04 [1]
 lazyeval         0.2.2      2019-03-15 [1]
 lifecycle        0.2.0      2020-03-06 [1]
 lpSolve          5.6.13.3   2019-08-19 [1]
 lubridate        1.7.4      2018-04-11 [1]
 magrittr       * 1.5        2014-11-22 [1]
 MASS             7.3-51.4   2019-03-31 [1]
 Matrix           1.2-17     2019-03-22 [1]
 memoise          1.1.0      2017-04-21 [1]
 mitools          2.4        2019-04-26 [1]
 modelr           0.1.4      2019-02-18 [1]
 mstate           0.2.11     2018-04-09 [1]
 muhaz            1.2.6.1    2019-01-26 [1]
 munsell          0.5.0      2018-06-12 [1]
 mvtnorm          1.0-11     2019-06-19 [1]
 network        * 1.16.0     2019-12-01 [1]
 networkDynamic * 0.10.0     2019-04-05 [1]
 nlme             3.1-140    2019-05-12 [1]
 pillar           1.4.6      2020-07-10 [1]
 pkgbuild         1.1.0      2020-07-13 [1]
 pkgconfig        2.0.3      2019-09-22 [1]
 pkgload          1.1.0      2020-05-29 [1]
 prettyunits      1.1.1      2020-01-24 [1]
 processx         3.4.3      2020-07-05 [1]
 ps               1.3.4      2020-08-11 [1]
 purrr          * 0.3.4      2020-04-17 [1]
 quadprog         1.5-7      2019-05-06 [1]
 R6               2.4.1      2019-11-12 [1]
 RColorBrewer   * 1.1-2      2014-12-07 [1]
 Rcpp             1.0.5      2020-07-06 [1]
 readr          * 1.3.1      2018-12-21 [1]
 readxl           1.3.1      2019-03-13 [1]
 remotes          2.2.0      2020-07-21 [1]
 rlang            0.4.7      2020-07-09 [1]
 rmarkdown        2.3        2020-06-18 [1]
 robustbase       0.93-5     2019-05-12 [1]
 rprojroot        1.3-2      2018-01-03 [1]
 rstudioapi       0.11       2020-02-07 [1]
 rvest            0.3.4      2019-05-15 [1]
 scales           1.1.1      2020-05-11 [1]
 sessioninfo      1.1.1      2018-11-05 [1]
 srvyr            0.4.0      2020-07-30 [1]
 statnet.common   4.3.0      2019-06-02 [1]
 stringi          1.4.6      2020-02-17 [1]
 stringr        * 1.4.0      2019-02-10 [1]
 survey           4.0        2020-04-03 [1]
 survival       * 2.44-1.1   2019-04-01 [1]
 survminer      * 0.4.5      2019-08-03 [1]
 survMisc         0.5.5      2018-07-05 [1]
 tergm          * 3.6.1      2019-06-12 [1]
 testthat         2.3.2      2020-03-02 [1]
 tibble         * 3.0.3      2020-07-10 [1]
 tidyr          * 1.1.1      2020-07-31 [1]
 tidyselect       1.1.0      2020-05-11 [1]
 tidyverse      * 1.2.1      2017-11-14 [1]
 trust            0.1-8      2020-01-10 [1]
 usethis        * 1.6.1      2020-04-29 [1]
 vctrs            0.3.2      2020-07-15 [1]
 viridisLite      0.3.0      2018-02-01 [1]
 webshot          0.5.1      2018-09-28 [1]
 withr            2.2.0      2020-04-20 [1]
 xfun             0.16       2020-07-24 [1]
 xml2             1.3.2      2020-04-23 [1]
 xtable           1.8-4      2019-04-21 [1]
 yaml             2.2.1      2020-02-01 [1]
 zoo              1.8-6      2019-05-28 [1]
 source                               
 CRAN (R 3.6.0)                       
 CRAN (R 3.6.0)                       
 CRAN (R 3.6.2)                       
 Github (rstudio/bookdown@f9cf1ac)    
 CRAN (R 3.6.0)                       
 CRAN (R 3.6.2)                       
 CRAN (R 3.6.0)                       
 CRAN (R 3.6.0)                       
 CRAN (R 3.6.0)                       
 CRAN (R 3.6.1)                       
 CRAN (R 3.6.0)                       
 CRAN (R 3.6.0)                       
 CRAN (R 3.6.0)                       
 CRAN (R 3.6.0)                       
 local                                
 CRAN (R 3.6.0)                       
 CRAN (R 3.6.0)                       
 CRAN (R 3.6.0)                       
 CRAN (R 3.6.2)                       
 CRAN (R 3.6.0)                       
 CRAN (R 3.6.0)                       
 CRAN (R 3.6.2)                       
 CRAN (R 3.6.2)                       
 CRAN (R 3.6.0)                       
 CRAN (R 3.6.0)                       
 CRAN (R 3.6.0)                       
 CRAN (R 3.6.0)                       
 CRAN (R 3.6.0)                       
 CRAN (R 3.6.0)                       
 CRAN (R 3.6.0)                       
 CRAN (R 3.6.0)                       
 CRAN (R 3.6.2)                       
 CRAN (R 3.6.0)                       
 CRAN (R 3.6.0)                       
 CRAN (R 3.6.2)                       
 CRAN (R 3.6.0)                       
 CRAN (R 3.6.0)                       
 CRAN (R 3.6.0)                       
 CRAN (R 3.6.2)                       
 CRAN (R 3.6.2)                       
 CRAN (R 3.6.0)                       
 CRAN (R 3.6.0)                       
 CRAN (R 3.6.0)                       
 CRAN (R 3.6.0)                       
 CRAN (R 3.6.0)                       
 CRAN (R 3.6.2)                       
 CRAN (R 3.6.2)                       
 Github (benmarwick/huskydown@a909835)
 CRAN (R 3.6.0)                       
 CRAN (R 3.6.2)                       
 CRAN (R 3.6.0)                       
 CRAN (R 3.6.0)                       
 CRAN (R 3.6.0)                       
 CRAN (R 3.6.2)                       
 CRAN (R 3.6.0)                       
 CRAN (R 3.6.1)                       
 CRAN (R 3.6.0)                       
 CRAN (R 3.6.0)                       
 CRAN (R 3.6.0)                       
 CRAN (R 3.6.0)                       
 CRAN (R 3.6.0)                       
 CRAN (R 3.6.1)                       
 CRAN (R 3.6.1)                       
 CRAN (R 3.6.0)                       
 CRAN (R 3.6.0)                       
 CRAN (R 3.6.0)                       
 CRAN (R 3.6.0)                       
 CRAN (R 3.6.0)                       
 CRAN (R 3.6.0)                       
 CRAN (R 3.6.0)                       
 CRAN (R 3.6.0)                       
 CRAN (R 3.6.0)                       
 CRAN (R 3.6.1)                       
 CRAN (R 3.6.2)                       
 CRAN (R 3.6.2)                       
 CRAN (R 3.6.0)                       
 CRAN (R 3.6.2)                       
 CRAN (R 3.6.0)                       
 CRAN (R 3.6.2)                       
 CRAN (R 3.6.2)                       
 CRAN (R 3.6.2)                       
 CRAN (R 3.6.0)                       
 CRAN (R 3.6.0)                       
 CRAN (R 3.6.0)                       
 CRAN (R 3.6.2)                       
 CRAN (R 3.6.0)                       
 CRAN (R 3.6.0)                       
 CRAN (R 3.6.2)                       
 CRAN (R 3.6.2)                       
 CRAN (R 3.6.2)                       
 CRAN (R 3.6.0)                       
 CRAN (R 3.6.0)                       
 CRAN (R 3.6.0)                       
 CRAN (R 3.6.0)                       
 CRAN (R 3.6.2)                       
 CRAN (R 3.6.0)                       
 CRAN (R 3.6.2)                       
 CRAN (R 3.6.0)                       
 CRAN (R 3.6.0)                       
 CRAN (R 3.6.0)                       
 CRAN (R 3.6.2)                       
 CRAN (R 3.6.0)                       
 CRAN (R 3.6.0)                       
 CRAN (R 3.6.0)                       
 CRAN (R 3.6.0)                       
 CRAN (R 3.6.0)                       
 CRAN (R 3.6.2)                       
 CRAN (R 3.6.2)                       
 CRAN (R 3.6.2)                       
 CRAN (R 3.6.0)                       
 CRAN (R 3.6.0)                       
 CRAN (R 3.6.2)                       
 CRAN (R 3.6.2)                       
 CRAN (R 3.6.0)                       
 CRAN (R 3.6.0)                       
 CRAN (R 3.6.2)                       
 CRAN (R 3.6.2)                       
 CRAN (R 3.6.2)                       
 CRAN (R 3.6.0)                       
 CRAN (R 3.6.0)                       
 CRAN (R 3.6.0)                       

[1] /Library/Frameworks/R.framework/Versions/3.6/Resources/library
\end{verbatim}
\backmatter

\chapter*{References}\label{references}
\addcontentsline{toc}{chapter}{References}

\markboth{References}{References}

\noindent

\setlength{\parindent}{-0.20in} \setlength{\leftskip}{0.20in}
\setlength{\parskip}{8pt}

\hypertarget{refs}{}
\hypertarget{ref-Armbruster2017}{}
Armbruster, B., Wang, L., \& Morris, M. (2017). Forward reachable sets:
Analytically derived properties of connected components for dynamic
networks. \emph{Network Science}, \emph{5}(3), 328--354.
\url{http://doi.org/10.1017/nws.2017.10}

\hypertarget{ref-Burington2010}{}
Burington, B., Hughes, J. P., Whittington, W. L. H., Stoner, B.,
Garnett, G., Aral, S. O., \& Holmes, K. K. (2010). Estimating duration
in partnership studies: Issues, methods and examples. \emph{Sexually
Transmitted Infections}, \emph{86}(2), 84--89.
\url{http://doi.org/10.1136/sti.2009.037960}

\hypertarget{ref-Carnegie2015}{}
Carnegie, N. B., Krivitsky, P. N., Hunter, D. R., \& Goodreau, S. M.
(2015). An Approximation Method for Improving Dynamic Network Model
Fitting. \emph{Journal of Computational and Graphical Statistics},
\emph{24}(2), 502--519.
\url{http://doi.org/10.1080/10618600.2014.903087}

\hypertarget{ref-Goodreau2017}{}
Goodreau, S. M., Rosenberg, E. S., Jenness, S. M., Luisi, N.,
Stansfield, S. E., Millett, G. A., \& Sullivan, P. S. (2017). Sources of
racial disparities in HIV prevalence in men who have sex with men in
Atlanta, GA, USA: a modelling study. \emph{The Lancet HIV}, \emph{4}(7),
e311--e320. \url{http://doi.org/10.1016/S2352-3018(17)30067-X}

\hypertarget{ref-Henningsen2011}{}
Henningsen, A., \& Toomet, O. (2011). MaxLik: A package for maximum
likelihood estimation in R. \emph{Computational Statistics},
\emph{26}(3), 443--458. \url{http://doi.org/10.1007/s00180-010-0217-1}

\hypertarget{ref-Krivitsky2014}{}
Krivitsky, P. N., \& Handcock, M. S. (2014). A separable model for
dynamic networks. \emph{Journal of the Royal Statistical Society. Series
B: Statistical Methodology}, \emph{76}(1), 29--46.
\url{http://doi.org/10.1111/rssb.12014}

\hypertarget{ref-Morris1997}{}
Morris, M., \& Kretzschmar, M. (1997). Concurrent partnerships and the
spread of HIV. \emph{AIDS}, \emph{11}(5), 641--648.
\url{http://doi.org/10.1097/00002030-199705000-00012}

\hypertarget{ref-Singer2006}{}
Singer, M. C., Erickson, P. I., Badiane, L., Diaz, R., Ortiz, D.,
Abraham, T., \& Nicolaysen, A. M. (2006). Syndemics, sex and the city:
Understanding sexually transmitted diseases in social and cultural
context. \emph{Social Science and Medicine}, \emph{63}(8), 2010--2021.
\url{http://doi.org/10.1016/j.socscimed.2006.05.012}
\end{document}
