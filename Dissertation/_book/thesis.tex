% From https://github.com/UWIT-IAM/UWThesis

\documentclass [11pt, proquest] {uwthesis}[2015/03/03]


% fix for pandoc 1.14
\providecommand{\tightlist}{%
  \setlength{\itemsep}{0pt}\setlength{\parskip}{0pt}}

\newtheorem{theorem}{Jibberish}

%% \bibliography{references}

\hyphenation{mar-gin-al-ia}

%
% ----- apply watermark to every page
% ----- change 'stamp' to 'nostamp'
%------ to omit watermark
%
\usepackage[nostamp]{draftwatermark}
% % Use the following to make modification
\SetWatermarkText{DRAFT}
\SetWatermarkLightness{0.95}

%% for the per mil symbol
\usepackage[nointegrals]{wasysym}

%% for copyright symbol
\usepackage{textcomp}

%% to allow to rotate pages to landscape
\usepackage{lscape}
%% to adjust table column width
\usepackage{tabularx}

% suppress bottom page numbers on first page of each chapter
% because they overlap with text
\usepackage{etoolbox}
\patchcmd{\chapter}{plain}{empty}{}{}

%% for more attractive tables
\usepackage{booktabs}
\usepackage{longtable}


\usepackage{graphicx}


% Double spacing, if you want it.
% \def\dsp{\def\baselinestretch{2.0}\large\normalsize}
% \dsp

% If the Grad. Division insists that the first paragraph of a section
% be indented (like the others), then include this line:
% \usepackage{indentfirst}

%%%%%%%%%%%%%%%%%%
% If you want to use "sections" to partition your thesis
% un-comment the following:
%
% \counterwithout{section}{chapter}
% \setsecnumdepth{subsubsection}
% \def\sectionmark#1{\markboth{#1}{#1}}
% \def\subsectionmark#1{\markboth{#1}{#1}}
% \renewcommand{\thesection}{\arabic{section}}
% \renewcommand{\thesubsection}{\thesection.\arabic{subsection}}
% \makeatletter
% \let\l@subsection\l@section
% \let\l@section\l@chapter
% \makeatother
%
% \renewcommand{\thetable}{\arabic{table}}
% \renewcommand{\thefigure}{\arabic{figure}}
%
%%%%%%%%%%%%%%%%%%


%% Stuff from https://github.com/suchow/Dissertate

% The following line would print the thesis in a postscript font

% \usepackage{natbib}
% \def\bibpreamble{\protect\addcontentsline{toc}{chapter}{Bibliography}}

\setcounter{tocdepth}{1} % Print the chapter and sections to the toc
% controls depth of table of contents (toc): 0 = chapter, 1 = section, 2 = subsection

\usepackage{biblatex}

\prelimpages

%% from thesisdown
% To pass between YAML and LaTeX the dollar signs are added by CII
\Title{Emily's Thesis Title}
\Author{Emily Pollock}
\Year{2021?}
\Program{Biological Anthropology}
\Chair{Steven M. Goodreau}{Title of my chair}{Biological Anthropology}
\Signature{person 1}
\Signature{person 2}
\Signature{person 3}

% commands and environments needed by pandoc snippets
% extracted from the output of `pandoc -s`
%% Make R markdown code chunks work
\usepackage{array}
\usepackage{amssymb,amsmath}
\usepackage{ifxetex,ifluatex}
\ifxetex
  \usepackage{fontspec,xltxtra,xunicode}
  \defaultfontfeatures{Mapping=tex-text,Scale=MatchLowercase}
\else
  \ifluatex
    \usepackage{fontspec}
    \defaultfontfeatures{Mapping=tex-text,Scale=MatchLowercase}
  \else
    \usepackage[utf8]{inputenc}
  \fi
\fi
\usepackage{color}
\usepackage{fancyvrb}
\DefineShortVerb[commandchars=\\\{\}]{\|}
\DefineVerbatimEnvironment{Highlighting}{Verbatim}{commandchars=\\\{\}}
% Add ',fontsize=\small' for more characters per line
\newenvironment{Shaded}{}{}
\newcommand{\KeywordTok}[1]{\textcolor[rgb]{0.00,0.44,0.13}{\textbf{{#1}}}}
\newcommand{\DataTypeTok}[1]{\textcolor[rgb]{0.56,0.13,0.00}{{#1}}}
\newcommand{\DecValTok}[1]{\textcolor[rgb]{0.25,0.63,0.44}{{#1}}}
\newcommand{\BaseNTok}[1]{\textcolor[rgb]{0.25,0.63,0.44}{{#1}}}
\newcommand{\FloatTok}[1]{\textcolor[rgb]{0.25,0.63,0.44}{{#1}}}
\newcommand{\CharTok}[1]{\textcolor[rgb]{0.25,0.44,0.63}{{#1}}}
\newcommand{\StringTok}[1]{\textcolor[rgb]{0.25,0.44,0.63}{{#1}}}
\newcommand{\CommentTok}[1]{\textcolor[rgb]{0.38,0.63,0.69}{\textit{{#1}}}}
\newcommand{\OtherTok}[1]{\textcolor[rgb]{0.00,0.44,0.13}{{#1}}}
\newcommand{\AlertTok}[1]{\textcolor[rgb]{1.00,0.00,0.00}{\textbf{{#1}}}}
\newcommand{\FunctionTok}[1]{\textcolor[rgb]{0.02,0.16,0.49}{{#1}}}
\newcommand{\RegionMarkerTok}[1]{{#1}}
\newcommand{\ErrorTok}[1]{\textcolor[rgb]{1.00,0.00,0.00}{\textbf{{#1}}}}
\newcommand{\NormalTok}[1]{{#1}}
\newcommand{\OperatorTok}[1]{\textcolor[rgb]{0.00,0.44,0.13}{\textbf{{#1}}}}
\newcommand{\BuiltInTok}[1]{\textcolor[rgb]{0.00,0.44,0.13}{\textbf{{#1}}}}
\newcommand{\ControlFlowTok}[1]{\textcolor[rgb]{0.00,0.44,0.13}{\textbf{{#1}}}}


\ifxetex
  \usepackage[setpagesize=false, % page size defined by xetex
              unicode=false, % unicode breaks when used with xetex
              xetex,
              colorlinks=true,
              linkcolor=blue]{hyperref}
\else
  \usepackage[unicode=true,
              colorlinks=true,
              linkcolor=blue]{hyperref}
\fi
\hypersetup{breaklinks=true, pdfborder={0 0 0}}
\setlength{\parindent}{0pt}
\setlength{\parskip}{6pt plus 2pt minus 1pt}
\setlength{\emergencystretch}{3em}  % prevent overfull lines
\setcounter{secnumdepth}{2} %% controls section numbering, e.g. 1 or 1.2, or 1.2.3

\begin{document}
\copyrightpage

\titlepage

\setcounter{page}{-1}
\abstract{``Here is my abstract''}

\tableofcontents
\listoffigures
\listoftables

\acknowledgments{``My acknowledgments''}

\dedication{\begin{center}``My dedication''\end{center}}

\textpages


\chapter*{Introduction}\label{introduction}
\addcontentsline{toc}{chapter}{Introduction}
\begin{itemize}
\tightlist
\item
  social sciency interest in networks \& disease transmission
\end{itemize}
Anthropologists have long recognized the importance of social
connections and behavioral variation among humans and our nonhuman
primate relatives. Indeed, the ability for us to participate in distinct
but potentially interlocking complex social networks has fueled our
evolution as a species and made our uniquely elaborate life possible.
Network analysis has often been utilized as a way to visually and
quantitatively represent these ties in order to understand their effects
on those connected to each other, from kinship, social support and
social capital, to the diffusion of information and transmission of
disease. These latter networks are crucially important to our
understanding of how human biosocial variation influences our health,
where the oft-beneficial complex social networks we maintain and
navigate every day can also put us at risk of exposure to infection.

In order to understand the complex patterns by which sexually
transmitted infections (STIs) are transmitted throughout populations, we
first need to understand the behavior of human relationships and how
these behaviors generate the dynamic sexual network across which these
types of infections can spread.

This work is guided by the theoretical framework of the human ecology of
infectious disease, the investigation of how human behavior, social
patterns, and built environments interact with the broader pathogen
environment to influence our health. Of particular interest is not just
aggregate behavior, but also how variation in individual behavior
influences social patterns and alters the landscape through which
diseases spread, particularly as this variation relates to biological
age. Syndemic theory will also be used as a guide to understand how
variation in behaviors and patterns act synergistically to increase
vulnerability and exacerbate existing health disparities of certain
population subgroups (Singer et al. 2006).(Singer et al., 2006)
\begin{itemize}
\tightlist
\item
  transition to a history of the evolution of epidemic models (ie. from
  compartmental where everything is basically independent and
  exponential through to ERGMs where formation can be quite elaborate
  but we've never spent much time thinking about dissolution)
\end{itemize}
the history of

from pinksy textbook on stochastic modeling:\\
``Stochastic processes are ways of quantifying the dynamic relationships
of sequences of random events. Stochastic models play an important role
in elucidating many areas of the natural and engineering sciences. They
can be used to analyze the variability inherent in biological and
medical processes, to deal with uncertainties affecting man- agerial
decisions and with the complexities of psychological and social
interactions, and to provide new perspectives, methodology, models, and
intuition to aid in other mathematical and statistical studies.''
\begin{itemize}
\item
  dynamic networks require information about relationship duration
\item
  why doing a bit better on dissolution/duration, especially by age,
  will be extra important for thinking about certain interventions for
  certain relatively short-lived infections (e.g.~partner services in
  chlamydia).
\item
  including age in dynamic models may sound straightforward but as we're
  going to see adds a surprising amount of complexity
\end{itemize}
\chapter{Survival Analysis of Relationship Duration}\label{surv}
\begin{itemize}
\tightlist
\item
  main goal here to understand where in the distribution we may not be
  capturing when we use models based on a memoryless process, and
  explore some ways to do better within the constraints of feasibility
  imposed by epidemic models
\item
  so we primarily use the exponential distribution in the SA analyses
\item
  discuss survey data and censoring / truncation issues
\item
  And then flesh out the details of the analyses with a clear narrative
  and you have the core of a chapter!
\end{itemize}
\textbf{Data}\\
National Survey of Family Growth, 2006-2015.\\
This survey has had several redesigns since its inception as a women's
marriage and fertility survey in 1973 but since 2006, the National
Survey of Family Growth (NSFG) has continuously interviewed both men and
women aged 15-44 and gathered data on family life, reproductive health,
marriage, divorce, and sexual history information. Particularly relevant
to this work are the questions relating to sexual behavior
(i.e.~frequency of sex and use of condoms) and a 12-month detailed
partnership history of up to three partners including demographic
information for these partners, start and ends dates of relationships,
and how many are currently active. Every few years the most recent data
are released as waves and include weights designed to make the data
nationally representative (``NSFG - National Survey of Family Growth
Homepage'' 2019). These data will be the sole source of empirical data
in Aims 1 and 2 and will be combined with epidemiological and biological
literature in Aim 3 to parameterize the full model. The primary focus
will be restricted to those individuals aged 15-29, but in some cases
the full dataset (ages 15-44) will be considered.

\textbf{Methods}\\
First the empirical relational duration data (using the start and end
dates of all reported relationships in NSFG) will be investigated using
histograms (overall and stratified by age category). Then, due to the
issues of right censoring and left truncation as a result of survey
design in NSFG, a reference survival curve will be constructed from the
empirical data using a Modified Kaplan-Meier model following (Burington
et al. 2010) and using the R package `survival'. Next, exploratory
parametric models will be estimated from the data (with corrections for
right-censoring and left-truncation) using a variety of distributions
(namely and latent mixture components) to gain intuition about the
underlying generative processes. Initial models will be covariate-free
(representing the effects of relationship duration only on the chances
of survival) and additional models will begin to examine the influence
of age on relationship duration, including (but not limited to) the
ego's age category, the reported current age difference between ego and
alter, and the ego's age category at the beginning of each reported
relationship. All models without latent components will be fit using the
R package `flexsurv' and the likelihood functions for all models with
latent components will be personally developed and models will be fit
using the maxLik package (Jackson 2016; Henningsen and Toomet 2011).

\chapter{Population Aging and Network Statistics}\label{nets}

needs better title

\chapter{Chlamydia, Acquired Immunity, \& Expedited Partner
Therapy?}\label{ept}

some text from diss proposal:

C. trachomatis is an obligate intracellular bacterium transmitted
through sexual contact among humans. Chlamydial infections are most
often asymptomatic. Untreated infections in women are an additional
public health concern because they can lead to a variety of sequalae
including pelvic inflammatory disease, scarring of ovaries and fallopian
tubes, ectopic pregnancies, chronic pain, and infertility. Repeat
infections are common and are an additional risk factor for the
development of the above sequelae (Brunham and Rey-Ladino 2005). There
is a great deal of uncertainty regarding the natural history of
chlamydia, but the duration of infection for untreated individuals is
generally thought to be up to 6 months for men and a year or more for
women (Golden et al. 2000; Satterwhite et al. 2013). Chlamydia is
usually treated with azithromycin or doxycycline, and unlike other
common STIs like syphilis and gonorrhea, true antibiotic resistance is
rare (Kong et al. 2015).

Chlamydia is the most common reportable disease in the United States and
incidence, particularly adolescents and young adults aged 15-29, is
increasing nationwide. The Centers for Disease Control and Prevention
(CDC) estimates that half of all new STI infections (including
gonorrhea, syphilis, and others) occur in those aged 15-24 despite them
making up only a quarter of the sexually active population. Even in
places like King County, Washington, where overall rates have remained
stable, longstanding acknowledged disparities in prevalence by race are
marked and continue to increase (2015 SKCPH STD Report). These rates are
particularly distressing in light of the fertility consequences of
long-term infection and reinfection: it is estimated that in King
County, over 60\% of non-Hispanic Black women have had at least one
chlamydia infection by age 34 (a rate 5x higher than non-Hispanic White
women) and 1 in 500 of non-Hispanic Black women develops
chlamydia-associated tubal factor infertility over their life-course
(Chambers et al. 2018).

The United States has some of the highest STD rates in the
industrialized world, and despite this, funding for public health
programs dedicated to these issues has largely declined (CDC 2016 STD
Report). As a result, few health departments are able to offer
traditional partner notification services, where a patient who tests
positive for an STI gives the contact information of their recent sex
partners to the health department, and the department then contacts
their partners with the hope that these partners will then get tested
and, if necessary, treated. Expedited partner therapy (EPT) was
developed with this scenario in mind (See figure 2). Under EPT, a
patient who tests positive, upon receipt of their own treatment,
receives either additional antibiotic pills for their recent sexual
partners or prescriptions for treatment that their partners can fill.
The patient then is expected to hand-deliver either the treatment or
prescription to their partner(s), who take the medicine at their own
discretion and without the need for a positive lab test. By using these
actors to essentially leverage their sexual network in reverse, this
system hopes to decrease the time to treatment for all possible infected
partners and increase the total number of partners treated. It can also
reduce re-infection among the index patients if the partnerships are
ongoing. There have been several clinical trials of EPT across the US
(and Europe), including Washington State. These trails demonstrated that
relative to traditional referral practices, EPT provision increased the
proportion of partners who were ultimately treated, reduced the number
of individuals who were re-infected at follow-up, and was less costly if
at least 30\% of partners were treated via EPT (CITE). Despite these
results and a growing body of evidence in support, widespread
implementation of EPT has been slow and there are still many questions
to be answered.

Annals of Internal Medicine Article High Incidence of New Sexually
Transmitted Infections in the Year following a Sexually Transmitted
Infection: A Case for Rescreening - Peterman et al

Arrested Immunity Hypothesis One of the paradoxes in era of modern
public health is that chlamydia incidence has actually increased overall
in the presence of mass control programs. In Sweden, Norway, Finland and
Canada the rates initially decreased but then resumed increasing, and in
Australia, United States, and the United Kingdom the rates never stopped
increasing even after program initiation, although this second pattern
has been attributed to the challenges of implementing control programs
consistently throughout a large population (Brunham and Rekart 2008).
These areas now experience incidence rates higher than rates prior to
introduction of control programs. Additionally, a regression analysis
using data from family planning clinics in Region X of the United States
(Alaska, Washington, Idaho, and Oregon) found that, after controlling
for any changes in demographics, sexual behaviors, and increased
sensitivity of clinical tests, there was a remaining 5\% `true' and
unexplained annual increase in chlamydia positivity from 1997-2004 (Fine
et al. 2008). In response to these and other examples of unabated
chlamydia infection in the presence of control programs, Brunham and
Reckart have proposed the arrested immunity hypothesis (Brunham and
Rekart 2008). Under this hypothesis, early detection and treatment of
chlamydia interrupts the development of acquired immunity, making
treated individuals particularly vulnerable to reinfection almost
immediately after treatment. While we have no natural history studies of
chlamydia infection in humans that address the development, duration,
and extent of immunity, there is growing evidence beyond rodent models
and trends in incidence that partial immunity can develop and play a
role. Rodent models of chlamydial infection suggest that a high
proportion are able to resolve their primary infection and are
temporarily resistant to infection. Rodents that then eventually become
reinfected with chlamydia have a shorter duration of disease, lower
pathogen load and decreased inflammatory response (Rank et al. 2003).
However, it has also been shown that treatment early in the course of
infection interrupts the development of this protective immunity (Su et
al. 2002). There is also some indirect evidence in humans. A 2010 review
article acknowledged that in several studies of infection status among
couples, the rates of discordance (i.e.~one partner is infected while
the other is not), are higher for chlamydia than for gonorrhea and that
this discordance increases with age, providing indirect evidence for
some level of protective immunity to chlamydia that increases with age,
likely due to exposure over time. There is little immunity that develops
to gonorrheal infection due to high levels of antigenic variation
(Batteiger et al. 2010). Recent modeling using data from both the UK and
United States has demonstrated that at least some immunity to chlamydia
following natural clearance is necessary to generate observed patterns
in incidence (Omori, Chemaitelly, and Abu-Raddad 2019). These questions
are particularly relevant in the context of expedited partner therapy,
where the goal is to interrupt transmission by treated individuals and
their partners as quickly as possible. However, due to the arrested
immunity of those treated quickly, if the timing of delivery and uptake
of partners is not sufficient, the initially treated is likely at higher
risk of reinfection than under the standard referral scenario. If
sufficient numbers of partners are treated effectively and quickly and
transmission throughout the network is greatly diminished, then EPT may
be able to overcome the effects of this arrested immunity.

\chapter*{Conclusion}\label{conclusion}
\addcontentsline{toc}{chapter}{Conclusion}

We conclude.

\appendix

\chapter{The First Appendix}\label{the-first-appendix}

additional figures?

\chapter{The Second Appendix}\label{the-second-appendix}

more technical stuff in here?

\chapter*{Colophon}\label{colophon}
\addcontentsline{toc}{chapter}{Colophon}

This document is set in \href{https://github.com/georgd/EB-Garamond}{EB
Garamond}, \href{https://github.com/adobe-fonts/source-code-pro/}{Source
Code Pro} and \href{http://www.latofonts.com/lato-free-fonts/}{Lato}.
The body text is set at 11pt with \(\familydefault\).

It was written in R Markdown and \(\LaTeX\), and rendered into PDF using
\href{https://github.com/benmarwick/huskydown}{huskydown} and
\href{https://github.com/rstudio/bookdown}{bookdown}.

This document was typeset using the XeTeX typesetting system, and the
\href{http://staff.washington.edu/fox/tex/}{University of Washington
Thesis class} class created by Jim Fox. Under the hood, the
\href{https://github.com/UWIT-IAM/UWThesis}{University of Washington
Thesis LaTeX template} is used to ensure that documents conform
precisely to submission standards. Other elements of the document
formatting source code have been taken from the
\href{https://github.com/stevenpollack/ucbthesis}{Latex, Knitr, and
RMarkdown templates for UC Berkeley's graduate thesis}, and
\href{https://github.com/suchow/Dissertate}{Dissertate: a LaTeX
dissertation template to support the production and typesetting of a PhD
dissertation at Harvard, Princeton, and NYU}

The source files for this thesis, along with all the data files, have
been organised into an R package, xxx, which is available at
\url{https://github.com/xxx/xxx}. A hard copy of the thesis can be found
in the University of Washington library.

This version of the thesis was generated on 2020-08-12 13:57:36. The
repository is currently at this commit:

The computational environment that was used to generate this version is
as follows:
\begin{verbatim}
- Session info ---------------------------------------------------------------
 setting  value                       
 version  R version 3.6.1 (2019-07-05)
 os       macOS Catalina 10.15.3      
 system   x86_64, darwin15.6.0        
 ui       X11                         
 language (EN)                        
 collate  en_US.UTF-8                 
 ctype    en_US.UTF-8                 
 tz       America/Los_Angeles         
 date     2020-08-12                  

- Packages -------------------------------------------------------------------
 package     * version date       lib source                               
 assertthat    0.2.1   2019-03-21 [1] CRAN (R 3.6.0)                       
 backports     1.1.8   2020-06-17 [1] CRAN (R 3.6.2)                       
 bookdown      0.20.2  2020-08-06 [1] Github (rstudio/bookdown@f9cf1ac)    
 callr         3.4.3   2020-03-28 [1] CRAN (R 3.6.2)                       
 cli           2.0.2   2020-02-28 [1] CRAN (R 3.6.0)                       
 colorspace    1.4-1   2019-03-18 [1] CRAN (R 3.6.0)                       
 crayon        1.3.4   2017-09-16 [1] CRAN (R 3.6.0)                       
 desc          1.2.0   2018-05-01 [1] CRAN (R 3.6.0)                       
 devtools    * 2.3.1   2020-07-21 [1] CRAN (R 3.6.2)                       
 digest        0.6.25  2020-02-23 [1] CRAN (R 3.6.0)                       
 dplyr         1.0.1   2020-07-31 [1] CRAN (R 3.6.2)                       
 ellipsis      0.3.1   2020-05-15 [1] CRAN (R 3.6.2)                       
 evaluate      0.14    2019-05-28 [1] CRAN (R 3.6.0)                       
 fansi         0.4.1   2020-01-08 [1] CRAN (R 3.6.0)                       
 fs            1.5.0   2020-07-31 [1] CRAN (R 3.6.2)                       
 generics      0.0.2   2018-11-29 [1] CRAN (R 3.6.0)                       
 ggplot2       3.3.2   2020-06-19 [1] CRAN (R 3.6.2)                       
 git2r         0.27.1  2020-05-03 [1] CRAN (R 3.6.2)                       
 glue          1.4.1   2020-05-13 [1] CRAN (R 3.6.2)                       
 gtable        0.3.0   2019-03-25 [1] CRAN (R 3.6.0)                       
 htmltools     0.5.0   2020-06-16 [1] CRAN (R 3.6.2)                       
 huskydown   * 0.0.5   2020-08-06 [1] Github (benmarwick/huskydown@a909835)
 knitr         1.29    2020-06-23 [1] CRAN (R 3.6.2)                       
 lifecycle     0.2.0   2020-03-06 [1] CRAN (R 3.6.0)                       
 magrittr      1.5     2014-11-22 [1] CRAN (R 3.6.0)                       
 memoise       1.1.0   2017-04-21 [1] CRAN (R 3.6.0)                       
 munsell       0.5.0   2018-06-12 [1] CRAN (R 3.6.0)                       
 pillar        1.4.6   2020-07-10 [1] CRAN (R 3.6.2)                       
 pkgbuild      1.1.0   2020-07-13 [1] CRAN (R 3.6.2)                       
 pkgconfig     2.0.3   2019-09-22 [1] CRAN (R 3.6.0)                       
 pkgload       1.1.0   2020-05-29 [1] CRAN (R 3.6.2)                       
 prettyunits   1.1.1   2020-01-24 [1] CRAN (R 3.6.0)                       
 processx      3.4.3   2020-07-05 [1] CRAN (R 3.6.2)                       
 ps            1.3.3   2020-05-08 [1] CRAN (R 3.6.2)                       
 purrr         0.3.4   2020-04-17 [1] CRAN (R 3.6.2)                       
 R6            2.4.1   2019-11-12 [1] CRAN (R 3.6.0)                       
 remotes       2.2.0   2020-07-21 [1] CRAN (R 3.6.2)                       
 rlang         0.4.7   2020-07-09 [1] CRAN (R 3.6.2)                       
 rmarkdown     2.3     2020-06-18 [1] CRAN (R 3.6.2)                       
 rprojroot     1.3-2   2018-01-03 [1] CRAN (R 3.6.0)                       
 rstudioapi    0.11    2020-02-07 [1] CRAN (R 3.6.0)                       
 scales        1.1.1   2020-05-11 [1] CRAN (R 3.6.2)                       
 sessioninfo   1.1.1   2018-11-05 [1] CRAN (R 3.6.0)                       
 stringi       1.4.6   2020-02-17 [1] CRAN (R 3.6.0)                       
 stringr       1.4.0   2019-02-10 [1] CRAN (R 3.6.0)                       
 testthat      2.3.2   2020-03-02 [1] CRAN (R 3.6.0)                       
 tibble        3.0.3   2020-07-10 [1] CRAN (R 3.6.2)                       
 tidyselect    1.1.0   2020-05-11 [1] CRAN (R 3.6.2)                       
 usethis     * 1.6.1   2020-04-29 [1] CRAN (R 3.6.2)                       
 vctrs         0.3.2   2020-07-15 [1] CRAN (R 3.6.2)                       
 withr         2.2.0   2020-04-20 [1] CRAN (R 3.6.2)                       
 xfun          0.16    2020-07-24 [1] CRAN (R 3.6.2)                       
 yaml          2.2.1   2020-02-01 [1] CRAN (R 3.6.0)                       

[1] /Library/Frameworks/R.framework/Versions/3.6/Resources/library
\end{verbatim}
\backmatter

\chapter*{References}\label{references}
\addcontentsline{toc}{chapter}{References}

\markboth{References}{References}

\noindent

\setlength{\parindent}{-0.20in} \setlength{\leftskip}{0.20in}
\setlength{\parskip}{8pt}

\hypertarget{refs}{}
\hypertarget{ref-Singer2006}{}
Singer, M. C., Erickson, P. I., Badiane, L., Diaz, R., Ortiz, D.,
Abraham, T., \& Nicolaysen, A. M. (2006). Syndemics, sex and the city:
Understanding sexually transmitted diseases in social and cultural
context. \emph{Social Science and Medicine}, \emph{63}(8), 2010--2021.
\url{http://doi.org/10.1016/j.socscimed.2006.05.012}
\end{document}
