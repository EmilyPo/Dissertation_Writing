% From https://github.com/UWIT-IAM/UWThesis

\documentclass [11pt, proquest] {uwthesis}[2015/03/03]


% fix for pandoc 1.14
\providecommand{\tightlist}{%
  \setlength{\itemsep}{0pt}\setlength{\parskip}{0pt}}

\newtheorem{theorem}{Jibberish}

%% \bibliography{references}

\hyphenation{mar-gin-al-ia}

%
% ----- apply watermark to every page
% ----- change 'stamp' to 'nostamp'
%------ to omit watermark
%
\usepackage[nostamp]{draftwatermark}
% % Use the following to make modification
\SetWatermarkText{DRAFT}
\SetWatermarkLightness{0.95}

%% for the per mil symbol
\usepackage[nointegrals]{wasysym}

%% for copyright symbol
\usepackage{textcomp}

%% to allow to rotate pages to landscape
\usepackage{lscape}
%% to adjust table column width
\usepackage{tabularx}

% suppress bottom page numbers on first page of each chapter
% because they overlap with text
\usepackage{etoolbox}
\patchcmd{\chapter}{plain}{empty}{}{}

%% for more attractive tables
\usepackage{booktabs}
\usepackage{longtable}


\usepackage{graphicx}


% Double spacing, if you want it.
% \def\dsp{\def\baselinestretch{2.0}\large\normalsize}
% \dsp

% If the Grad. Division insists that the first paragraph of a section
% be indented (like the others), then include this line:
% \usepackage{indentfirst}

%%%%%%%%%%%%%%%%%%
% If you want to use "sections" to partition your thesis
% un-comment the following:
%
% \counterwithout{section}{chapter}
% \setsecnumdepth{subsubsection}
% \def\sectionmark#1{\markboth{#1}{#1}}
% \def\subsectionmark#1{\markboth{#1}{#1}}
% \renewcommand{\thesection}{\arabic{section}}
% \renewcommand{\thesubsection}{\thesection.\arabic{subsection}}
% \makeatletter
% \let\l@subsection\l@section
% \let\l@section\l@chapter
% \makeatother
%
% \renewcommand{\thetable}{\arabic{table}}
% \renewcommand{\thefigure}{\arabic{figure}}
%
%%%%%%%%%%%%%%%%%%


%% Stuff from https://github.com/suchow/Dissertate

% The following line would print the thesis in a postscript font

% \usepackage{natbib}
% \def\bibpreamble{\protect\addcontentsline{toc}{chapter}{Bibliography}}

\setcounter{tocdepth}{1} % Print the chapter and sections to the toc
% controls depth of table of contents (toc): 0 = chapter, 1 = section, 2 = subsection

\usepackage{biblatex}

\prelimpages

%% from thesisdown
% To pass between YAML and LaTeX the dollar signs are added by CII
\Title{Emily's Thesis Title}
\Author{Emily Pollock}
\Year{2021?}
\Program{Biological Anthropology}
\Chair{Steven M. Goodreau}{Title of my chair}{Biological Anthropology}
\Signature{person 1}
\Signature{person 2}
\Signature{person 3}

% commands and environments needed by pandoc snippets
% extracted from the output of `pandoc -s`
%% Make R markdown code chunks work
\usepackage{array}
\usepackage{amssymb,amsmath}
\usepackage{ifxetex,ifluatex}
\ifxetex
  \usepackage{fontspec,xltxtra,xunicode}
  \defaultfontfeatures{Mapping=tex-text,Scale=MatchLowercase}
\else
  \ifluatex
    \usepackage{fontspec}
    \defaultfontfeatures{Mapping=tex-text,Scale=MatchLowercase}
  \else
    \usepackage[utf8]{inputenc}
  \fi
\fi
\usepackage{color}
\usepackage{fancyvrb}
\DefineShortVerb[commandchars=\\\{\}]{\|}
\DefineVerbatimEnvironment{Highlighting}{Verbatim}{commandchars=\\\{\}}
% Add ',fontsize=\small' for more characters per line
\newenvironment{Shaded}{}{}
\newcommand{\KeywordTok}[1]{\textcolor[rgb]{0.00,0.44,0.13}{\textbf{{#1}}}}
\newcommand{\DataTypeTok}[1]{\textcolor[rgb]{0.56,0.13,0.00}{{#1}}}
\newcommand{\DecValTok}[1]{\textcolor[rgb]{0.25,0.63,0.44}{{#1}}}
\newcommand{\BaseNTok}[1]{\textcolor[rgb]{0.25,0.63,0.44}{{#1}}}
\newcommand{\FloatTok}[1]{\textcolor[rgb]{0.25,0.63,0.44}{{#1}}}
\newcommand{\CharTok}[1]{\textcolor[rgb]{0.25,0.44,0.63}{{#1}}}
\newcommand{\StringTok}[1]{\textcolor[rgb]{0.25,0.44,0.63}{{#1}}}
\newcommand{\CommentTok}[1]{\textcolor[rgb]{0.38,0.63,0.69}{\textit{{#1}}}}
\newcommand{\OtherTok}[1]{\textcolor[rgb]{0.00,0.44,0.13}{{#1}}}
\newcommand{\AlertTok}[1]{\textcolor[rgb]{1.00,0.00,0.00}{\textbf{{#1}}}}
\newcommand{\FunctionTok}[1]{\textcolor[rgb]{0.02,0.16,0.49}{{#1}}}
\newcommand{\RegionMarkerTok}[1]{{#1}}
\newcommand{\ErrorTok}[1]{\textcolor[rgb]{1.00,0.00,0.00}{\textbf{{#1}}}}
\newcommand{\NormalTok}[1]{{#1}}
\newcommand{\OperatorTok}[1]{\textcolor[rgb]{0.00,0.44,0.13}{\textbf{{#1}}}}
\newcommand{\BuiltInTok}[1]{\textcolor[rgb]{0.00,0.44,0.13}{\textbf{{#1}}}}
\newcommand{\ControlFlowTok}[1]{\textcolor[rgb]{0.00,0.44,0.13}{\textbf{{#1}}}}


\ifxetex
  \usepackage[setpagesize=false, % page size defined by xetex
              unicode=false, % unicode breaks when used with xetex
              xetex,
              colorlinks=true,
              linkcolor=blue]{hyperref}
\else
  \usepackage[unicode=true,
              colorlinks=true,
              linkcolor=blue]{hyperref}
\fi
\hypersetup{breaklinks=true, pdfborder={0 0 0}}
\setlength{\parindent}{0pt}
\setlength{\parskip}{6pt plus 2pt minus 1pt}
\setlength{\emergencystretch}{3em}  % prevent overfull lines
\setcounter{secnumdepth}{2} %% controls section numbering, e.g. 1 or 1.2, or 1.2.3

\begin{document}
\copyrightpage

\titlepage

\setcounter{page}{-1}
\abstract{``Here is my abstract''}

\tableofcontents
\listoffigures
\listoftables

\acknowledgments{``My acknowledgments''}

\dedication{\begin{center}``My dedication''\end{center}}

\textpages


\chapter*{Introduction}\label{introduction}
\addcontentsline{toc}{chapter}{Introduction}

Anthropologists have long recognized the importance of social
connections and behavioral variation among humans and our nonhuman
primate relatives. Indeed, the ability for us to participate in distinct
but potentially interlocking complex social networks has fueled our
evolution as a species and made our uniquely elaborate life possible.
Network analysis has often been utilized as a way to visually and
quantitatively represent these ties in order to understand their effects
on those connected to each other, from kinship, social support and
social capital, to the diffusion of information and transmission of
disease. These latter networks are crucially important to our
understanding of how human biosocial variation influences our health,
where the oft-beneficial complex social networks we maintain and
navigate every day can also put us at risk of exposure to infection.

transition to STIs

In order to understand the complex patterns by which sexually
transmitted infections (STIs) are transmitted throughout populations, we
first need to understand the behavior of human relationships and how
these behaviors generate the dynamic sexual network across which these
types of infections can spread.

This work is guided by the theoretical framework of the human ecology of
infectious disease, the investigation of how human behavior, social
patterns, and built environments interact with the broader pathogen
environment to influence our health. Of particular interest is not just
aggregate behavior, but also how variation in individual behavior
influences social patterns and alters the landscape through which
diseases spread, particularly as this variation relates to biological
age. Syndemic theory will also be used as a guide to understand how
variation in behaviors and patterns act synergistically to increase
vulnerability and exacerbate existing health disparities of certain
population subgroups (Singer et al., 2006).

can I pull some stuff from my PAA abstract about age?
\begin{itemize}
\tightlist
\item
  transition to a history of the evolution of epidemic models (ie. from
  compartmental where everything is basically independent and
  exponential through to ERGMs where formation can be quite elaborate
  but we've never spent much time thinking about dissolution)
\end{itemize}
Mathematical models are quantitative representations of real-life
systems and the processes within these systems important to the outcome
of interest. This form of inquiry is particularly useful when classic
scientific experiments to understand disease spread or intervention
efficacy cannot be conducted for either practical or ethical reasons, or
when specific processes or parameter values in a system are unknown. In
these situations, we use mathematical modeling as an in-silico
laboratory to explore ideas and test hypotheses. Of course, the form and
complexity of these models are determined by a variety of factors
including the type of question that needs answering and the natural
history of the infection of interest, but many types of mathematical
models rely on similar underlying assumptions. Without diving too deep
into the history of epidemic modeling, here I give a brief overview of
the various model forms to highlight some key similarities and
differences.

Initial mathematical models for epidemics were deterministic and
compartmental in nature. They did not represent people individually,
rather they group them into homogenous compartments representing
specific states of interest, a portion of which transitioned between
compartments at each time step based on a rate. In the most basic
models, the compartments are usually ``susceptible'' and ``infected''
and the rate of transition from susceptible to infected depends on the
rate of contact between the groups and the size of the infected group
relative to the whole population. Additional complexity can be added by
adding more compartments or states, like breaking down the state of
susceptible and infected into demographic states like race or age
groups, adding compartments for vector populations like mosquitoes, or
by representing a more complex natural history of the pathogen by
including states for groups such as ``exposed but not infectious'',
``recovered'', ``infected and symptomatic'', or ``infected and
asymptomatic'' to name a few. These models were deterministic in nature
because the transitions between compartments rely on unchanging rates:
the same proportion of each component transitions at each time point and
if you run a deterministic compartmental model (DCM) multiple times you
will alway have the same result.

Stochastic models grew out of this original framework as a way to
capture variability and uncertainty in the systems we wish to study. In
this scenario, some or all transitions between states were based on a
\emph{probability} of transitioning rather than a set rate, meaning that
not the same proportion of a state transitioned at every time step, but
on \emph{average}

Notice the assumptions implicit in the way transitions occur in these
models - it is memoryless, generating an exponential distribution (or
geometric if using discrete time).
\begin{itemize}
\item
  dynamic networks require information about relationship duration
\item
  why doing a bit better on dissolution/duration, especially by age,
  will be extra important for thinking about certain interventions for
  certain relatively short-lived infections (e.g.~partner services in
  chlamydia).
\item
  additionally, while births and deaths have been a part of models, only
  recently are we adding explicit age-dependent formation terms -- and
  age changes over the simulation -- what is this effect?\\
\item
  including age in dynamic models may sound straightforward but as we're
  going to see adds a surprising amount of complexity
\end{itemize}
\textbf{Notes from NME intro to modeling}

exponential -- memoryless survival function, exchangeability

all models have three basic components - elements (actors), states
(attributes), transitions (rates of movements between states)

DCMs within each compartment, people are homogenous flows are
represented by rates - the fraction of the aggregate count that moves
from one compartment to another any any time point each run gives the
same results\\
at the most basic level - what do you need to transmit infection? one
infected and one susceptible and some assumption(s) about the
transmission process (acts, transmissibility, etc)

``contacts'' somewhat misleading because multiple acts with same person
makes a huge difference - but DCMs treat all contacts as independent,
and/or contacts estimated as a full complete partnership

move into stochastic modeling (can be have stochastic and deterministic
components)- rates become probabilities

\chapter{Survival Analysis of Relationship Duration}\label{surv}
\begin{itemize}
\tightlist
\item
  main goal here to understand where in the distribution we may not be
  capturing when we use models based on a memoryless process, and
  explore some ways to do better within the constraints of feasibility
  imposed by epidemic models
\item
  so we primarily use the exponential distribution in the SA analyses
\item
  discuss survey data and censoring / truncation issues
\item
  And then flesh out the details of the analyses with a clear narrative
  and you have the core of a chapter!
\end{itemize}
The duration of sexual relationships across a population generates the
network structure largely responsible for either exposing individuals to
or protecting individuals from sexually transmitted infections (STIs).
In addition to dictating this period of possible exposure, relationship
durations relative to the pathogen-specific duration of infection are an
important driver of how quickly STIs can spread throughout a population.
Transmission beyond a pair of actors for infections with short durations
relative to relationship lengths is challenging and slow, and it is more
likely that an infection will be detected and treated or resolved
naturally prior to the dissolution of the relationship. If the duration
of infection and duration of relationships are more equal, there is a
greater chance that the infection can spread to future partners and
throughout the network. When partnerships overlap, transmission pathways
increase even among those individuals with few lifetime partners, and
this effect is even greater when the duration of overlap is large
(Armbruster, Wang, and Morris 2017; Morris and Kretzschmar 1997).

The pattern of relationship durations across the life-course is also
important because STIs often have distinct age patterns in terms of
prevalence. Individual age is often used as a predictor for risky sexual
behavior, but there is additional complexity when considering the effect
of age on the duration of relationships across the life-course. Young
age likely influences the immediate intentions for relationships
(i.e.~serious or casual), and the frequency at which individuals form
new relationships, but somewhat paradoxically it is also true that the
only people who can report extremely long relationships are those who
started them at young ages. This also introduces complex sampling issues
because most data on relationship durations is collected
cross-sectionally or retrospectively -- not longitudinally. Given the
importance of relationship duration to features of STI epidemiology
discussed above, there is growing interest in improving the
representation of relational durations in dynamic network models used to
study epidemics. This study demonstrates the ways in which the current
literature fails to represent this distribution and proposes a new
modeling framework to better capture these relationships across the
life-course.

One common class of models used to understand network influences on
patterns of STI transmission is known as separable temporal
exponential-family random graph models (STERGMs). These models are
governed by two expressions: one that represents the set of processes
that influence the formation of relationships, and a comparable one for
dissolution (Krivitsky and Handcock 2014). The current standard practice
for the dissolution models in this modeling framework assumes that once
a relationship begins, its persistence is governed by a constant hazard.
This memoryless process is a convenient simplifying assumption, adopted
because most hypotheses being explored relate to processes impacting
network formation or cross-sectional structure. However, it is unlikely
that this assumption faithfully represents the distribution all
relationship durations we observe across a wide range of ages.

Several recent models have begun to address this issue by splitting out
relationships into two categories: the first, marriages and
cohabitations or main partnerships, and the second, persistent or casual
partnerships. These are then modeled as separate networks
simultaneously. By structuring the model in this fashion, each network
has a hazard of dissolution specific to its type. (These models often
have a third network for one-time sexual contacts which last only one
time-step, but this network is not the focus of our study). While these
models are indeed able to reproduce the mean relationship lengths drawn
from empirical data, it remains unknown how well these strategies
reproduce the full distribution of lengths observed. In particular, the
memoryless assumption means that the modal length of main partnerships
remains near zero across all ages, which basic intuition says is not
true and descriptive data analysis confirms. Other work has considered
disaggregating relational durations by a single demographic attribute of
their members related to a hypothesis or prevention modality being
explored, but again with no further effort to capture the full
distribution, particularly by age (Goodreau et al. 2017; Jenness et al.
2017).

In this ongoing study, we seek to understand the changing distribution
of relationship duration over the life-course using data from the
National Survey of Family Growth, to evaluate which features the above
dissolution assumptions are capable of replicating and which they
cannot. We then introduce an alternative framework designed to more
faithfully represent these data and the different demographic and
data-collection processes that impact them in an age-structured
population over time. We use tools from both event history analysis and
network analysis to answer the following questions: First, under what
circumstances, if any, can an exponentially distributed time-to-event
model reasonably approximate empirical relationship duration data?
Second, does it make sense to lump marriages and cohabitations into one
network with the same dissolution probability? And third, can we better
capture the age-wise relationship distribution by using one network
where (1) relationships can transition between states (e.g.~from a
cohabitation into a marriage) rather than modeling several types
separately, and (2) where relational formation probabilities depend on
current relational status.

\textbf{Data}\\
The empirical data used in this study are drawn from the 2006-2010 and
2011-2015 waves of the National Survey of Family Growth (NSFG). The NSFG
surveys men and women aged 15-44 on many aspects of family life,
including but not limited to marriage and divorce, pregnancy,
contraception use, infertility, and other aspects of sexual and
reproductive behavior. In addition to the demographic information
recorded for each respondent and their sampling weights, in this study
we use the data collected in section C of the public use files on each
respondent's recent sexual partnerships with opposite-sex partners in
the last year, with a maximum of three partnerships reported. These data
include the century-month of first sexual contact, the century-month of
last sexual contact, whether the respondent considers this sexual
partnership to be ongoing, and the partnership status (marriage,
cohabitation, or other). We limit the combined data set to those
respondents who report at least one partnership in the last year. Out of
the original 43,303 respondents, our subset contains 32,516 respondents
who report on 40,443 sexual partnerships. Due to the study design, all
relationships that respondents report as ongoing on the day of interview
have right-censored relationship lengths, and there is left-truncation
present due to the large number or relationships that started prior to
the observation window but continued into it.

\textbf{Methods}\\
First the empirical relational duration data (using the start and end
dates of all reported relationships in NSFG) will be investigated using
histograms (overall and stratified by age category). Then, due to the
issues of right censoring and left truncation as a result of survey
design in NSFG, a reference survival curve will be constructed from the
empirical data using a Modified Kaplan-Meier model following (Burington
et al. 2010) and using the R package `survival'. Next, exploratory
parametric models will be estimated from the data (with corrections for
right-censoring and left-truncation) using a variety of distributions
(namely and latent mixture components) to gain intuition about the
underlying generative processes. Initial models will be covariate-free
(representing the effects of relationship duration only on the chances
of survival) and additional models will begin to examine the influence
of age on relationship duration, including (but not limited to) the
ego's age category, the reported current age difference between ego and
alter, and the ego's age category at the beginning of each reported
relationship. All models without latent components will be fit using the
R package `flexsurv' and the likelihood functions for all models with
latent components will be personally developed and models will be fit
using the maxLik package (Jackson 2016; Henningsen and Toomet 2011).

From PAA abstract: In our preliminary work, we first checked the
assumption that relationship duration can be modeled by a simple
memoryless process, and then explored some natural extensions to this
framework. In order to generate the reference distribution, we fit a
Kaplan-Meier model using a modified estimator to account for both
right-censoring and left-truncation following Burington et al (2010). We
then fit several parametric models (all adjusted for the above sampling
issues): first a simple exponential model to represent the memoryless
process assumption, then a Weibull distribution and Gamma distribution,
all with and without additional covariate attributes. Model fit was
evaluated primarily using the Akaike Information Criterion (AIC) and
visuals to understand which relationship lengths were represented better
than others, given our ultimate goal of adapting these into dynamic
network simulations. Below is a selection of explored models; parametric
models with covariate categories are displayed in color, with their
Kaplan-Meier reference curves plotted in black. All parameters in these
fitted models are statistically significant (p \textless{} 0.001).

\textbf{Results}

The first takeaway is that an exponential distribution alone is not
sufficient to capture the relationship distribution -- it overestimates
the survival of short relationships and underestimates the survival of
long relationships (top left figure, below). The Weibull and Gamma
perform better and capture more of the short relationships, suggesting
that there is important heterogeneity in the data, but like the first
models they also fail to capture the longest relationships. The age
category of the reporting individual is not explanatory across all age
categories (top right). This is perhaps not surprising, in that the age
distribution of relationship lengths is at least partly an emergent
property rather than a causal one. That is, no individual can have a
relationship that has lasted longer than they have been sexually active,
so the range of relationship lengths for young age categories is
relatively small. Meanwhile, the older age categories are challenging to
represent because the possible range of relationships is so much larger,
and are likely influenced not only by dissolution probabilities but also
by the changing formation probabilities over the lifecourse -- that is,
older people in long-term relationships do not start new relationships
at the same rate as others, and thus have relatively few relationships
that are short.

The next two models test how appropriate it is to group relationships
defined as marriages and cohabitations into the same dissolution model,
as has been done in recent STERGMs. We see clear evidence that marriages
and cohabitations have distinct hazards of dissolution and the combined
marriage and cohabitation curve, like the simple exponential for all
relationships, dramatically fails to capture both the shortest and
longest relationships of these types (bottom right and bottom left
figures, respectively. These results are similar to other work in family
demography that has shown significant differences in the risk of
dissolution between cohabitations and marriages due to variation in
joint lifestyles (van Houdt and Poortman 2018). These results suggest to
us that previously developed STERGM dissolution models that only capture
the mean relationship length are not appropriate approximations of the
data, and that cohabitation represents a distinctly separate type of
relationship from marriages and other casual relationships and should be
treated as such in our networks.

\chapter{Demography and Dynamic Network Simulations}\label{nets}

needs better title

Having gained insights about factors important (and not important) to
the patterns of relationship length over the age course from survival
analysis, the next steps initially seemed straightforward. First, I was
going to build a two-network simulation model comparable to recently
published models (where the casual/shorter relationships are represented
on one network and marriages and cohabitations are represented on
another) and analyze the patterns of relationship duration across the
simulated age range to understand the ways in which we are able to
recreate the empirical distribution and the ways in which we are not.
Second, I was going to build a network model with a new structure:
instead of modeling relationships on separate networks, I would begin
all relationships as casual relationships and have them transition over
time into cohabitations and marriages. Relationship dissolution
probability, as in the first model, would be based on relationship type.
By transitioning relationships over time -- a process much closer to
reality - instead of classifying certain relationships as, say,
marriages, at their onset, I hoped to match certain features of the
empirical distribution better. In particular: the increasingly uniform
distribution of relationship lengths at older ages as some individuals
maintain long-lasting marriages and others maintain cohabitations or
begin entirely new relationships.

Suffice it to say that I did not get to step two.

In mathematical models, the choice of model terms depends on the
question of interest and the underlying patterns in the data and this is
no less true for network models of sexual partnerships developed to
understand disease transmission. Several previously published models
using ERGMs and EpiModel to simulate epidemics focused on men who have
sex with men (MSM) populations in a narrow age range, 18-35 (\emph{cite
papers and also double check that this is true}). These models focused
on terms related to mixing patterns between races, the propensity to
form relationships with individuals relatively close in age, and the
likelihood of concurrent partnerships. Because prevalence of both main
and casual relationships remained relatively stable over the small age
range, the models did not include terms that used age as a predictor of
relationship formation. However, in this project, we focus on
heterosexual relationships over a larger age range (15-45). Unlike MSM,
there are large clear differences in the prevalence of main and casual
partnerships over this age range (insert figure), so we will need to
include terms that include age in our model. In addition to influencing
the distribution of relationship duration, these differences are likely
to be especially important if we want to use this type of model to
understand the processes that generate the large observed differences in
bacterial STI prevalence by age - originally one of the broader goals of
this dissertation.
\begin{itemize}
\tightlist
\item
  insert figure of mean degree by age (with unrestricted alters)
\end{itemize}
As it turns out, adding age-related formation terms and other important
demographic processes to a dynamic network simulation has some
unexpected consequences.

\section{Model Overview \& STERGM fit}\label{model-overview-stergm-fit}

several general trends in relationship formation (finish write-up and
cite) --
\begin{itemize}
\tightlist
\item
  individuals often select partners that are not their exact age
\item
  this difference in partner ages often increases over the life course
  (i.e.~adults usually have wider age differences between their partners
  than do adolescents)
\item
  it is common for men to partner with younger women (although the sex
  differences in relationship formation are not explored in this model,
  it's important to note that in a more realistic model the effect of
  aging out would disproportionately affect the women whose partners age
  out before them
\end{itemize}
The terms in the model are relatively simplistic so I did not expect to
hit the degree-by-age distribution exactly, but the aging process within
the network simulation and artificial node death at age 45, when many
nodes are in relationships, heavily influences the degree distribution
at the tail end of the age range in several ways.

(include model terms and coefs and explain terms) (full description of
EpiModelHIV module flow w/ parameters in appendix, brief overview here)

\section{Diagnostic Results}\label{diagnostic-results}

(to demonstrate closed-system effectiveness, assumes fixed nodal
attributes)

The final step in evaluate the performance of an estimated STERGM prior
to the simulation is to run a dynamic diagnostic. In this diagnostic, we
simulate the STERGM for X repetitions of Y time steps and evaluate the
network statistics over time. At each time step, ties can form and ties
can dissolve based on the model coefficients. If the model is estimated
properly and sufficient MCMC intervals are used, the network formation
statistics should hover around their estimated targets. In this
diagnostic we also evaluate the duration of ties and the rate of tie
dissolution to ensure the dissolution targets are met. It is important
to note that this diagnostic is an indicator of model performance in a
closed system: all nodal attributes are fixed, no nodes exit, and no new
nodes enter the population.

\section{Overview of Demographic
Processes}\label{overview-of-demographic-processes}

The simulations run using the EpiModel API are distinct from the above
dynamic diagnostic in that in addition to tie formation and dissolution
at every time step, a series of modules is run that govern important
demographic processes: node departure, node entry, aging, and sexual
debut. Nodes automatically depart the model at age 45. This boundary was
determined by two things: 1) individuals this point contribute almost a
negligible amount of the yearly bacterial STI incidence {[}CDC figure{]}
and 2) the National Survey of Family Growth, the empirical data from
which we estimate our model, only surveys adults aged 15-45. There are
likely other sources of information that we could use to increase the
age range, but it did not seem necessary to our questions of interest.
Note that implicit in this decision is the elimination of all reported
relationships among egos aged 15-45 whose \emph{partners} are outside of
this age range. The degree distribution that we actually use to estimate
the model (and are trying to maintain during simulation) looks rather
different than the original distribution shown above, particularly in
the marriage/cohabitation network. {[}insert figure{]}. We will consider
the consequences of effect a later section. In addition to the age
boundary at 45, all individuals experience the possibility of dying at
each time step. Each node belongs to a class based on their
5-year-age-category and their sex, and is evaluated for death at every
time step with the probability determined by data from published in U.S.
Vital Statistics documents (cite). Given that our age range is
relatively young, departures due to background mortality are uncommon
relative to the effect of the age boundary on which nodes depart the
model. Nodes enter at age 15 at a rate based on the expected number of
departures per time step in order to keep the population size relatively
stable. Like ASMR, the actual number of entires per time step is
stochastic but maintains a population size within 1-2\% of the starting
size of 50,000 nodes. (Do I need to explain why we want to keep the pop
stable?). Each time step in the simulation represents one week, so nodes
age by 1/52 per time step. The sexual debut process is somewhat trickier
to estimate and dynamically represent.
\begin{itemize}
\tightlist
\item
  Sexual Debut\\
\item
  dynamic process, nodal attribute not necessarily monotonic in
  cross-sectional data - gets into period/cohort stuff that is
  interesting but not addressed here
\item
  debut vs ``eligibility'' and what information we need for model vs
  what we have in the data now, in this model setup, the rate of sexual
  debut does not influence the birth/arrival rate in the model - as
  mentioned above the model is designed to have a relatively stable
  population with an arrival rate based on the expected number of
  departures at each time step. Sexual debut however does dictate
  whether an individual is allowed to form a relationship, and the
  number of un-debut persons is jointly estimated in the model, so
  deviations from the original distribution will influence the
  likelihood of tie formation\ldots{}
\end{itemize}
the underlying population structure is not particularly
complex\ldots{}so it's not the act of aging (or migration etc) that
generates problems but the fact that age is so tied to the probability
of having (or not having) a certain type of relationship.

\section{Original Simulation}\label{original-simulation}

Narrative order:\\
1. Cross network terms - we're going to avoid them due to
complications\\
2. Older Partner\\
* offset for older partner\\
* keeping people in\\
* conclusion: we keep offset in all future scenarios but not older
partners\\
3. Sexual Debut\\
* Debut\\
* eligibility\\
* conclusion: debut not eligibility\\
4. let's think about why we're seeing the things we are\\
* both networks have too few edges, particularly in early years\\
* both dissolution rates slightly too low\\
* marriage/cohab network: duration far too low\\
* casual network: duration too high\\
* tests:\\
* marriage/cohab -- adj formation for earlier edges and longer
durations\\
* casual -- adj formation for earlier edges but also departure for too
long relationships
\begin{verbatim}
Edapprox  
\end{verbatim}
\begin{enumerate}
\def\labelenumi{\arabic{enumi}.}
\setcounter{enumi}{4}
\item
\end{enumerate}
asides / future work\\
* cross-network terms - probably going to do most analysis on the
independent networks but will show both and point at where there are
holes (hey by the time this gets finished maybe Chad will have already
figured this out)\\
* what distribution of formation terms / debut parameters will generate
the desired mean degree by age distribution\\
* need to think about race and sex differences in formation and
absdiff(age) by sex if we want to use this for applied work

\chapter{Chlamydia, Acquired Immunity, \& Expedited Partner
Therapy?}\label{ept}

copying over some text from diss proposal:

C. trachomatis is an obligate intracellular bacterium transmitted
through sexual contact among humans. Chlamydial infections are most
often asymptomatic. Untreated infections in women are an additional
public health concern because they can lead to a variety of sequalae
including pelvic inflammatory disease, scarring of ovaries and fallopian
tubes, ectopic pregnancies, chronic pain, and infertility. Repeat
infections are common and are an additional risk factor for the
development of the above sequelae (Brunham and Rey-Ladino 2005). There
is a great deal of uncertainty regarding the natural history of
chlamydia, but the duration of infection for untreated individuals is
generally thought to be up to 6 months for men and a year or more for
women (Golden et al. 2000; Satterwhite et al. 2013). Chlamydia is
usually treated with azithromycin or doxycycline, and unlike other
common STIs like syphilis and gonorrhea, true antibiotic resistance is
rare (Kong et al. 2015).

Chlamydia is the most common reportable disease in the United States and
incidence, particularly adolescents and young adults aged 15-29, is
increasing nationwide. The Centers for Disease Control and Prevention
(CDC) estimates that half of all new STI infections (including
gonorrhea, syphilis, and others) occur in those aged 15-24 despite them
making up only a quarter of the sexually active population. Even in
places like King County, Washington, where overall rates have remained
stable, longstanding acknowledged disparities in prevalence by race are
marked and continue to increase (2015 SKCPH STD Report). These rates are
particularly distressing in light of the fertility consequences of
long-term infection and reinfection: it is estimated that in King
County, over 60\% of non-Hispanic Black women have had at least one
chlamydia infection by age 34 (a rate 5x higher than non-Hispanic White
women) and 1 in 500 of non-Hispanic Black women develops
chlamydia-associated tubal factor infertility over their life-course
(Chambers et al. 2018).

The United States has some of the highest STD rates in the
industrialized world, and despite this, funding for public health
programs dedicated to these issues has largely declined (CDC 2016 STD
Report). As a result, few health departments are able to offer
traditional partner notification services, where a patient who tests
positive for an STI gives the contact information of their recent sex
partners to the health department, and the department then contacts
their partners with the hope that these partners will then get tested
and, if necessary, treated. Expedited partner therapy (EPT) was
developed with this scenario in mind (See figure 2). Under EPT, a
patient who tests positive, upon receipt of their own treatment,
receives either additional antibiotic pills for their recent sexual
partners or prescriptions for treatment that their partners can fill.
The patient then is expected to hand-deliver either the treatment or
prescription to their partner(s), who take the medicine at their own
discretion and without the need for a positive lab test. By using these
actors to essentially leverage their sexual network in reverse, this
system hopes to decrease the time to treatment for all possible infected
partners and increase the total number of partners treated. It can also
reduce re-infection among the index patients if the partnerships are
ongoing. There have been several clinical trials of EPT across the US
(and Europe), including Washington State. These trails demonstrated that
relative to traditional referral practices, EPT provision increased the
proportion of partners who were ultimately treated, reduced the number
of individuals who were re-infected at follow-up, and was less costly if
at least 30\% of partners were treated via EPT (CITE). Despite these
results and a growing body of evidence in support, widespread
implementation of EPT has been slow and there are still many questions
to be answered.

Annals of Internal Medicine Article High Incidence of New Sexually
Transmitted Infections in the Year following a Sexually Transmitted
Infection: A Case for Rescreening - Peterman et al

Arrested Immunity Hypothesis One of the paradoxes in era of modern
public health is that chlamydia incidence has actually increased overall
in the presence of mass control programs. In Sweden, Norway, Finland and
Canada the rates initially decreased but then resumed increasing, and in
Australia, United States, and the United Kingdom the rates never stopped
increasing even after program initiation, although this second pattern
has been attributed to the challenges of implementing control programs
consistently throughout a large population (Brunham and Rekart 2008).
These areas now experience incidence rates higher than rates prior to
introduction of control programs. Additionally, a regression analysis
using data from family planning clinics in Region X of the United States
(Alaska, Washington, Idaho, and Oregon) found that, after controlling
for any changes in demographics, sexual behaviors, and increased
sensitivity of clinical tests, there was a remaining 5\% `true' and
unexplained annual increase in chlamydia positivity from 1997-2004 (Fine
et al. 2008). In response to these and other examples of unabated
chlamydia infection in the presence of control programs, Brunham and
Reckart have proposed the arrested immunity hypothesis (Brunham and
Rekart 2008). Under this hypothesis, early detection and treatment of
chlamydia interrupts the development of acquired immunity, making
treated individuals particularly vulnerable to reinfection almost
immediately after treatment. While we have no natural history studies of
chlamydia infection in humans that address the development, duration,
and extent of immunity, there is growing evidence beyond rodent models
and trends in incidence that partial immunity can develop and play a
role. Rodent models of chlamydial infection suggest that a high
proportion are able to resolve their primary infection and are
temporarily resistant to infection. Rodents that then eventually become
reinfected with chlamydia have a shorter duration of disease, lower
pathogen load and decreased inflammatory response (Rank et al. 2003).
However, it has also been shown that treatment early in the course of
infection interrupts the development of this protective immunity (Su et
al. 2002). There is also some indirect evidence in humans. A 2010 review
article acknowledged that in several studies of infection status among
couples, the rates of discordance (i.e.~one partner is infected while
the other is not), are higher for chlamydia than for gonorrhea and that
this discordance increases with age, providing indirect evidence for
some level of protective immunity to chlamydia that increases with age,
likely due to exposure over time. There is little immunity that develops
to gonorrheal infection due to high levels of antigenic variation
(Batteiger et al. 2010). Recent modeling using data from both the UK and
United States has demonstrated that at least some immunity to chlamydia
following natural clearance is necessary to generate observed patterns
in incidence (Omori, Chemaitelly, and Abu-Raddad 2019). These questions
are particularly relevant in the context of expedited partner therapy,
where the goal is to interrupt transmission by treated individuals and
their partners as quickly as possible. However, due to the arrested
immunity of those treated quickly, if the timing of delivery and uptake
of partners is not sufficient, the initially treated is likely at higher
risk of reinfection than under the standard referral scenario. If
sufficient numbers of partners are treated effectively and quickly and
transmission throughout the network is greatly diminished, then EPT may
be able to overcome the effects of this arrested immunity.

\chapter*{Conclusion}\label{conclusion}
\addcontentsline{toc}{chapter}{Conclusion}

We conclude.

\appendix

\chapter{The First Appendix}\label{the-first-appendix}

additional figures?

\chapter{The Second Appendix}\label{the-second-appendix}

more technical stuff in here?

\chapter*{Colophon}\label{colophon}
\addcontentsline{toc}{chapter}{Colophon}

This document is set in \href{https://github.com/georgd/EB-Garamond}{EB
Garamond}, \href{https://github.com/adobe-fonts/source-code-pro/}{Source
Code Pro} and \href{http://www.latofonts.com/lato-free-fonts/}{Lato}.
The body text is set at 11pt with \(\familydefault\).

It was written in R Markdown and \(\LaTeX\), and rendered into PDF using
\href{https://github.com/benmarwick/huskydown}{huskydown} and
\href{https://github.com/rstudio/bookdown}{bookdown}.

This document was typeset using the XeTeX typesetting system, and the
\href{http://staff.washington.edu/fox/tex/}{University of Washington
Thesis class} class created by Jim Fox. Under the hood, the
\href{https://github.com/UWIT-IAM/UWThesis}{University of Washington
Thesis LaTeX template} is used to ensure that documents conform
precisely to submission standards. Other elements of the document
formatting source code have been taken from the
\href{https://github.com/stevenpollack/ucbthesis}{Latex, Knitr, and
RMarkdown templates for UC Berkeley's graduate thesis}, and
\href{https://github.com/suchow/Dissertate}{Dissertate: a LaTeX
dissertation template to support the production and typesetting of a PhD
dissertation at Harvard, Princeton, and NYU}

The source files for this thesis, along with all the data files, have
been organised into an R package, xxx, which is available at
\url{https://github.com/xxx/xxx}. A hard copy of the thesis can be found
in the University of Washington library.

This version of the thesis was generated on 2020-08-26 10:25:30. The
repository is currently at this commit:

The computational environment that was used to generate this version is
as follows:
\begin{verbatim}
- Session info ---------------------------------------------------------------
 setting  value                       
 version  R version 3.6.1 (2019-07-05)
 os       macOS Catalina 10.15.3      
 system   x86_64, darwin15.6.0        
 ui       X11                         
 language (EN)                        
 collate  en_US.UTF-8                 
 ctype    en_US.UTF-8                 
 tz       America/Los_Angeles         
 date     2020-08-26                  

- Packages -------------------------------------------------------------------
 package     * version date       lib source                               
 assertthat    0.2.1   2019-03-21 [1] CRAN (R 3.6.0)                       
 backports     1.1.9   2020-08-24 [1] CRAN (R 3.6.2)                       
 bookdown      0.20.2  2020-08-06 [1] Github (rstudio/bookdown@f9cf1ac)    
 callr         3.4.3   2020-03-28 [1] CRAN (R 3.6.2)                       
 cli           2.0.2   2020-02-28 [1] CRAN (R 3.6.0)                       
 colorspace    1.4-1   2019-03-18 [1] CRAN (R 3.6.0)                       
 crayon        1.3.4   2017-09-16 [1] CRAN (R 3.6.0)                       
 desc          1.2.0   2018-05-01 [1] CRAN (R 3.6.0)                       
 devtools    * 2.3.1   2020-07-21 [1] CRAN (R 3.6.2)                       
 digest        0.6.25  2020-02-23 [1] CRAN (R 3.6.0)                       
 dplyr         1.0.2   2020-08-18 [1] CRAN (R 3.6.2)                       
 ellipsis      0.3.1   2020-05-15 [1] CRAN (R 3.6.2)                       
 evaluate      0.14    2019-05-28 [1] CRAN (R 3.6.0)                       
 fansi         0.4.1   2020-01-08 [1] CRAN (R 3.6.0)                       
 fs            1.5.0   2020-07-31 [1] CRAN (R 3.6.2)                       
 generics      0.0.2   2018-11-29 [1] CRAN (R 3.6.0)                       
 ggplot2       3.3.2   2020-06-19 [1] CRAN (R 3.6.2)                       
 git2r         0.27.1  2020-05-03 [1] CRAN (R 3.6.2)                       
 glue          1.4.1   2020-05-13 [1] CRAN (R 3.6.2)                       
 gtable        0.3.0   2019-03-25 [1] CRAN (R 3.6.0)                       
 htmltools     0.5.0   2020-06-16 [1] CRAN (R 3.6.2)                       
 huskydown   * 0.0.5   2020-08-06 [1] Github (benmarwick/huskydown@a909835)
 knitr         1.29    2020-06-23 [1] CRAN (R 3.6.2)                       
 lifecycle     0.2.0   2020-03-06 [1] CRAN (R 3.6.0)                       
 magrittr      1.5     2014-11-22 [1] CRAN (R 3.6.0)                       
 memoise       1.1.0   2017-04-21 [1] CRAN (R 3.6.0)                       
 munsell       0.5.0   2018-06-12 [1] CRAN (R 3.6.0)                       
 pillar        1.4.6   2020-07-10 [1] CRAN (R 3.6.2)                       
 pkgbuild      1.1.0   2020-07-13 [1] CRAN (R 3.6.2)                       
 pkgconfig     2.0.3   2019-09-22 [1] CRAN (R 3.6.0)                       
 pkgload       1.1.0   2020-05-29 [1] CRAN (R 3.6.2)                       
 prettyunits   1.1.1   2020-01-24 [1] CRAN (R 3.6.0)                       
 processx      3.4.3   2020-07-05 [1] CRAN (R 3.6.2)                       
 ps            1.3.4   2020-08-11 [1] CRAN (R 3.6.2)                       
 purrr         0.3.4   2020-04-17 [1] CRAN (R 3.6.2)                       
 R6            2.4.1   2019-11-12 [1] CRAN (R 3.6.0)                       
 remotes       2.2.0   2020-07-21 [1] CRAN (R 3.6.2)                       
 rlang         0.4.7   2020-07-09 [1] CRAN (R 3.6.2)                       
 rmarkdown     2.3     2020-06-18 [1] CRAN (R 3.6.2)                       
 rprojroot     1.3-2   2018-01-03 [1] CRAN (R 3.6.0)                       
 rstudioapi    0.11    2020-02-07 [1] CRAN (R 3.6.0)                       
 scales        1.1.1   2020-05-11 [1] CRAN (R 3.6.2)                       
 sessioninfo   1.1.1   2018-11-05 [1] CRAN (R 3.6.0)                       
 stringi       1.4.6   2020-02-17 [1] CRAN (R 3.6.0)                       
 stringr       1.4.0   2019-02-10 [1] CRAN (R 3.6.0)                       
 testthat      2.3.2   2020-03-02 [1] CRAN (R 3.6.0)                       
 tibble        3.0.3   2020-07-10 [1] CRAN (R 3.6.2)                       
 tidyselect    1.1.0   2020-05-11 [1] CRAN (R 3.6.2)                       
 usethis     * 1.6.1   2020-04-29 [1] CRAN (R 3.6.2)                       
 vctrs         0.3.2   2020-07-15 [1] CRAN (R 3.6.2)                       
 withr         2.2.0   2020-04-20 [1] CRAN (R 3.6.2)                       
 xfun          0.16    2020-07-24 [1] CRAN (R 3.6.2)                       
 yaml          2.2.1   2020-02-01 [1] CRAN (R 3.6.0)                       

[1] /Library/Frameworks/R.framework/Versions/3.6/Resources/library
\end{verbatim}
\backmatter

\chapter*{References}\label{references}
\addcontentsline{toc}{chapter}{References}

\markboth{References}{References}

\noindent

\setlength{\parindent}{-0.20in} \setlength{\leftskip}{0.20in}
\setlength{\parskip}{8pt}

\hypertarget{refs}{}
\hypertarget{ref-Singer2006}{}
Singer, M. C., Erickson, P. I., Badiane, L., Diaz, R., Ortiz, D.,
Abraham, T., \& Nicolaysen, A. M. (2006). Syndemics, sex and the city:
Understanding sexually transmitted diseases in social and cultural
context. \emph{Social Science and Medicine}, \emph{63}(8), 2010--2021.
\url{http://doi.org/10.1016/j.socscimed.2006.05.012}
\end{document}
